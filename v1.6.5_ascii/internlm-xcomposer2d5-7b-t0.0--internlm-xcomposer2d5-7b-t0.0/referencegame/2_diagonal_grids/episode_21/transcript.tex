
\documentclass{article}
\usepackage{colortbl}
\usepackage{makecell}
\usepackage{multirow}
\usepackage{supertabular}

\begin{document}

\newcounter{utterance}

\twocolumn

{ \footnotesize  \setcounter{utterance}{1}
\setlength{\tabcolsep}{0pt}
\begin{supertabular}{c@{$\;$}|p{.15\linewidth}@{}p{.15\linewidth}p{.15\linewidth}p{.15\linewidth}p{.15\linewidth}p{.15\linewidth}}

    \# & $\;$A & \multicolumn{4}{c}{Game Master} & $\;\:$B\\
    \hline 

    \theutterance \stepcounter{utterance}  

    & & \multicolumn{4}{p{0.6\linewidth}}{\cellcolor[rgb]{0.9,0.9,0.9}{%
	\makecell[{{p{\linewidth}}}]{% 
	  \tt {\tiny [A$\langle$GM]}  
	 You are given three grids, where each of them is 5 by 5 in size.\\ \tt Grids have empty cells marked with "▢" and filled cells marked with "X".\\ \tt Your task is to generate a referring expression that best describes the target grid while distinguishing it from the two other distractor grids.\\ \tt The first grid is the target grid, and the following two grids are the distractors.\\ \tt \\ \tt Target grid:\\ \tt \\ \tt ▢ ▢ ▢ ▢ X\\ \tt ▢ ▢ ▢ X X\\ \tt ▢ ▢ X X X\\ \tt ▢ X X X X\\ \tt X X X X X\\ \tt \\ \tt Distractor grid 1:\\ \tt \\ \tt ▢ ▢ ▢ ▢ X\\ \tt ▢ ▢ ▢ X ▢\\ \tt ▢ ▢ X ▢ ▢\\ \tt ▢ X ▢ ▢ ▢\\ \tt X ▢ ▢ ▢ ▢\\ \tt \\ \tt Distractor grid 2:\\ \tt \\ \tt X ▢ ▢ ▢ X\\ \tt ▢ X ▢ X ▢\\ \tt ▢ ▢ X ▢ ▢\\ \tt ▢ X ▢ X ▢\\ \tt X ▢ ▢ ▢ X\\ \tt \\ \tt Instruction: Describe the target grid.\\ \tt Generate the referring expression starting with the tag "Expression: " for the given target grid. Omit any other text. 
	  } 
	   } 
	   } 
	 & \\ 
 

    \theutterance \stepcounter{utterance}  

    & \multicolumn{4}{p{0.6\linewidth}}{\cellcolor[rgb]{0.8,1,0.9}{%
	\makecell[{{p{\linewidth}}}]{% 
	  \tt {\tiny [A$\rangle$GM]}  
	 To describe the target grid while distinguishing it from the distractor grids, we can use the following referring expression:\\ \tt \\ \tt Expression: Target Grid\\ \tt - In the first row, there are three empty cells followed by an "X" at the end.\\ \tt - The second row starts with an empty cell, followed by two empty cells, and ends with two "X" cells.\\ \tt - The third row has an empty cell at the beginning, followed by two "X" cells, and ends with an "X" cell.\\ \tt - The fourth row is entirely filled with "X" cells.\\ \tt - The fifth row starts with two "X" cells, followed by an empty cell, and ends with two "X" cells.\\ \tt \\ \tt This description captures the unique arrangement of filled and empty cells in the target grid, distinguishing it from the distractor grids which have different patterns of filled and empty cells. 
	  } 
	   } 
	   } 
	 & & \\ 
 

    \theutterance \stepcounter{utterance}  

    & & & \multicolumn{2}{p{0.3\linewidth}}{\cellcolor[rgb]{0.95,0.95,0.95}{%
	\makecell[{{p{\linewidth}}}]{% 
	  \tt {\tiny [GM$|$GM]}  
	 Invalid generated expression 
	  } 
	   } 
	   } 
	 & & \\ 
 

\end{supertabular}
}

\end{document}
