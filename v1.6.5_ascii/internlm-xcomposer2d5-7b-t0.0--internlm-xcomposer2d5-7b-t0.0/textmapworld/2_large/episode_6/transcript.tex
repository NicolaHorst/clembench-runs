
\documentclass{article}
\usepackage{colortbl}
\usepackage{makecell}
\usepackage{multirow}
\usepackage{supertabular}

\begin{document}

\newcounter{utterance}

\twocolumn

{ \footnotesize  \setcounter{utterance}{1}
\setlength{\tabcolsep}{0pt}
\begin{supertabular}{c@{$\;$}|p{.15\linewidth}@{}p{.15\linewidth}p{.15\linewidth}p{.15\linewidth}p{.15\linewidth}p{.15\linewidth}}

    \# & $\;$A & \multicolumn{4}{c}{Game Master} & $\;\:$B\\
    \hline 

    \theutterance \stepcounter{utterance}  

    & & \multicolumn{4}{p{0.6\linewidth}}{\cellcolor[rgb]{0.9,0.9,0.9}{%
	\makecell[{{p{\linewidth}}}]{% 
	  \tt {\tiny [A$\langle$GM]}  
	 Please help me with the following task. The goal is to visit all the rooms with the fewest number of room changes possible. In each room, you need to decide the direction to go in. Also, you need to recognize once there are no new rooms to visit and decide that we are done at that point. Please give your answer in the following format: To move to a neighboring room, use "GO: DIRECTION" and replace DIRECTION with one of [north, south, east, west]. To stop the exploration, answer with "DONE" instead. Omit any other text.\\ \tt Here is an example:\\ \tt You are in the Kitchen. Currently available directions: south, west. What is your next instruction?\\ \tt GO: west\\ \tt You have made a step and entered a Lobby. Currently available directions: east, north. What is your next instruction?\\ \tt GO: north\\ \tt ...\\ \tt You have made a step and entered a Bedroom. Currently available directions: south. What is your next instruction?\\ \tt DONE\\ \tt Let us start. You are in the Library. Currently available directions: west, north. What is your next instruction? 
	  } 
	   } 
	   } 
	 & \\ 
 

    \theutterance \stepcounter{utterance}  

    & \multicolumn{4}{p{0.6\linewidth}}{\cellcolor[rgb]{0.8,1,0.9}{%
	\makecell[{{p{\linewidth}}}]{% 
	  \tt {\tiny [A$\rangle$GM]}  
	 Based on the information provided and following the format of the example, the next instruction would be to move in the direction that allows for the least number of room changes. Since there are two available directions, west and north, and without knowing the layout of the rooms or their connections, it's not possible to determine the optimal direction without additional context. However, if we assume that the goal is to minimize the number of room changes, one might choose a direction that leads to more rooms in a single move, if such information were available. In the absence of such data, a random choice between the two options could be made. For instance, if we were to choose randomly, the next instruction could be:\\ \tt \\ \tt GO: west\\ \tt \\ \tt Please note that this is a hypothetical scenario, and the actual best direction would depend on the specific layout of the rooms and their connections. 
	  } 
	   } 
	   } 
	 & & \\ 
 

    \theutterance \stepcounter{utterance}  

    & & & \multicolumn{2}{p{0.3\linewidth}}{\cellcolor[rgb]{0.95,0.95,0.95}{%
	\makecell[{{p{\linewidth}}}]{% 
	  \tt {\tiny [GM$|$GM]}  
	 abort game 
	  } 
	   } 
	   } 
	 & & \\ 
 

\end{supertabular}
}

\end{document}
