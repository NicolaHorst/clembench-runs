
\documentclass{article}
\usepackage{colortbl}
\usepackage{makecell}
\usepackage{multirow}
\usepackage{supertabular}

\begin{document}

\newcounter{utterance}

\twocolumn

{ \footnotesize  \setcounter{utterance}{1}
\setlength{\tabcolsep}{0pt}
\begin{supertabular}{c@{$\;$}|p{.15\linewidth}@{}p{.15\linewidth}p{.15\linewidth}p{.15\linewidth}p{.15\linewidth}p{.15\linewidth}}

    \# & $\;$A & \multicolumn{4}{c}{Game Master} & $\;\:$B\\
    \hline 

    \theutterance \stepcounter{utterance}  

    & & \multicolumn{4}{p{0.6\linewidth}}{\cellcolor[rgb]{0.9,0.9,0.9}{%
	\makecell[{{p{\linewidth}}}]{% 
	  \tt {\tiny [A$\langle$GM]}  
	 {'role': 'user', 'content': 'You are given three images, one is called target and the other two are distractors.\nYour task is to generate a referring expression that best describes the target image while distinguishing it from the two other distractor images.\nThe first image is the target, the second image is a distractor, and the third image is a distractor. Instruction: Describe the target image. Generate the referring expression starting with the tag "Expression: " for the given target image. Omit any other text.', 'image': ['games/multimodal_referencegame/resources/grid_images/59.png', 'games/multimodal_referencegame/resources/grid_images/53.png', 'games/multimodal_referencegame/resources/grid_images/58.png']} 
	  } 
	   } 
	   } 
	 & \\ 
 

    \theutterance \stepcounter{utterance}  

    & \multicolumn{4}{p{0.6\linewidth}}{\cellcolor[rgb]{0.8,1,0.9}{%
	\makecell[{{p{\linewidth}}}]{% 
	  \tt {\tiny [A$\rangle$GM]}  
	 Upon careful examination of the image, the target image and the two distractors share the following characteristics: the figure in the center of the image, the color palette, and the stylistic approach. However, the distinguishing features that differentiate the target image from the distractors lie in specific details such as subtle differences in the facial features, the hair style, and minor variations in the clothing patterns. Additionally, the lighting and shadows in the target image may slightly differ from the distractors, contributing to its unique identification.\\ \tt \\ \tt To describe the target image within the context of the task, one could approach it by focusing on these unique elements that set it apart from the distractors. For instance:\\ \tt "Unique details set apart the target image—such as distinct facial features, hair style variations, and clothing patterns. Subtle lighting and shadow differences contribute to its distinct appearance."\\ \tt \\ \tt This description aims to provide a clear and specific differentiation that serves the purpose of distinguishing the target image in the context of the given task. 
	  } 
	   } 
	   } 
	 & & \\ 
 

    \theutterance \stepcounter{utterance}  

    & & & \multicolumn{2}{p{0.3\linewidth}}{\cellcolor[rgb]{0.95,0.95,0.95}{%
	\makecell[{{p{\linewidth}}}]{% 
	  \tt {\tiny [GM$|$GM]}  
	 Invalid generated expression 
	  } 
	   } 
	   } 
	 & & \\ 
 

\end{supertabular}
}

\end{document}
