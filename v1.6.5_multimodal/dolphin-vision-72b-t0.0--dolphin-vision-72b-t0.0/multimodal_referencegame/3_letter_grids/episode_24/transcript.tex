
\documentclass{article}
\usepackage{colortbl}
\usepackage{makecell}
\usepackage{multirow}
\usepackage{supertabular}

\begin{document}

\newcounter{utterance}

\twocolumn

{ \footnotesize  \setcounter{utterance}{1}
\setlength{\tabcolsep}{0pt}
\begin{supertabular}{c@{$\;$}|p{.15\linewidth}@{}p{.15\linewidth}p{.15\linewidth}p{.15\linewidth}p{.15\linewidth}p{.15\linewidth}}

    \# & $\;$A & \multicolumn{4}{c}{Game Master} & $\;\:$B\\
    \hline 

    \theutterance \stepcounter{utterance}  

    & & \multicolumn{4}{p{0.6\linewidth}}{\cellcolor[rgb]{0.9,0.9,0.9}{%
	\makecell[{{p{\linewidth}}}]{% 
	  \tt {\tiny [A$\langle$GM]}  
	 {'role': 'user', 'content': 'You are given three images, one is called target and the other two are distractors.\nYour task is to generate a referring expression that best describes the target image while distinguishing it from the two other distractor images.\nThe first image is the target, the second image is a distractor, and the third image is a distractor. Instruction: Describe the target image. Generate the referring expression starting with the tag "Expression: " for the given target image. Omit any other text.', 'image': ['games/multimodal_referencegame/resources/grid_images/38.png', 'games/multimodal_referencegame/resources/grid_images/37.png', 'games/multimodal_referencegame/resources/grid_images/33.png']} 
	  } 
	   } 
	   } 
	 & \\ 
 

    \theutterance \stepcounter{utterance}  

    & \multicolumn{4}{p{0.6\linewidth}}{\cellcolor[rgb]{0.8,1,0.9}{%
	\makecell[{{p{\linewidth}}}]{% 
	  \tt {\tiny [A$\rangle$GM]}  
	 Expression: A grid with four X-shaped marks aligned in a central vertical column. 
	  } 
	   } 
	   } 
	 & & \\ 
 

    \theutterance \stepcounter{utterance}  

    & & & \multicolumn{2}{p{0.3\linewidth}}{\cellcolor[rgb]{0.95,0.95,0.95}{%
	\makecell[{{p{\linewidth}}}]{% 
	  \tt {\tiny [GM$|$GM]}  
	 Expression: A grid with four X-shaped marks aligned in a central vertical column. 
	  } 
	   } 
	   } 
	 & & \\ 
 

    \theutterance \stepcounter{utterance}  

    & & \multicolumn{4}{p{0.6\linewidth}}{\cellcolor[rgb]{0.9,0.9,0.9}{%
	\makecell[{{p{\linewidth}}}]{% 
	  \tt {\tiny [GM$\rangle$B]}  
	 {'role': 'user', 'content': 'You are given three images.\nYou are also given a referring expression that describes one of the given images.\nYour task is to select the image that matches the given referring expression.\nGenerate only the number (in text) of the image that matches the given expression by selecting first, second, or third.\n\nExpression: A grid with four X-shaped marks aligned in a central vertical column.\nQuestion: Which image does the expression refer to?\nStart with the tag "Answer: ", followed by your selection. Omit any other text.\n', 'image': ['games/multimodal_referencegame/resources/grid_images/38.png', 'games/multimodal_referencegame/resources/grid_images/37.png', 'games/multimodal_referencegame/resources/grid_images/33.png']} 
	  } 
	   } 
	   } 
	 & \\ 
 

    \theutterance \stepcounter{utterance}  

    & & & \multicolumn{4}{p{0.6\linewidth}}{\cellcolor[rgb]{1,0.85,0.72}{%
	\makecell[{{p{\linewidth}}}]{% 
	  \tt {\tiny [GM$\langle$B]}  
	 Answer: The referring expression describes an image with a grid and four X-shaped marks aligned in the central column of the grid. By examining the provided image, it is clear that there are five squares with X-shaped marks, but they do not form a vertical column; instead, they are spread across the grid. Therefore, None of these options accurately describe the image provided. If the expression were to match accurately, the image would contain only four Xs aligned vertically in the center of the grid, which is not the case here. Hence, the appropriate response to the given question is that none of the images match the referring expression. 
	  } 
	   } 
	   } 
	  \\ 
 

    \theutterance \stepcounter{utterance}  

    & & & \multicolumn{2}{p{0.3\linewidth}}{\cellcolor[rgb]{0.95,0.95,0.95}{%
	\makecell[{{p{\linewidth}}}]{% 
	  \tt {\tiny [GM$|$GM]}  
	 Invalid generated choice 
	  } 
	   } 
	   } 
	 & & \\ 
 

\end{supertabular}
}

\end{document}
