
\documentclass{article}
\usepackage{colortbl}
\usepackage{makecell}
\usepackage{multirow}
\usepackage{supertabular}

\begin{document}

\newcounter{utterance}

\twocolumn

{ \footnotesize  \setcounter{utterance}{1}
\setlength{\tabcolsep}{0pt}
\begin{supertabular}{c@{$\;$}|p{.15\linewidth}@{}p{.15\linewidth}p{.15\linewidth}p{.15\linewidth}p{.15\linewidth}p{.15\linewidth}}

    \# & $\;$A & \multicolumn{4}{c}{Game Master} & $\;\:$B\\
    \hline 

    \theutterance \stepcounter{utterance}  

    & & & \multicolumn{2}{p{0.3\linewidth}}{\cellcolor[rgb]{0.95,0.95,0.95}{%
	\makecell[{{p{\linewidth}}}]{% 
	  \tt {\tiny [GM$|$GM]}  
	 {"image": ["games/mm_mapworld/resources/images/ADE_train_00007362.jpg"]} 
	  } 
	   } 
	   } 
	 & & \\ 
 

    \theutterance \stepcounter{utterance}  

    & & \multicolumn{4}{p{0.6\linewidth}}{\cellcolor[rgb]{0.9,0.9,0.9}{%
	\makecell[{{p{\linewidth}}}]{% 
	  \tt {\tiny [A$\langle$GM]}  
	 {"start": [2, 0], "size": 8, "game": "mm_mapworld_graphs"} 
	  } 
	   } 
	   } 
	 & \\ 
 

    \theutterance \stepcounter{utterance}  

    & \multicolumn{4}{p{0.6\linewidth}}{\cellcolor[rgb]{0.8,1,0.9}{%
	\makecell[{{p{\linewidth}}}]{% 
	  \tt {\tiny [A$\rangle$GM]}  
	 We are currently in this room. Please help me with the following task. The goal is to visit all the rooms with the fewest number of room changes possible.  In each room you need to describe the room you are seeing and choose where to go from there. Additionally, you need to provide a graph representing the map you have uncovered. Also, you need to recognize once there are no new rooms to visit and decide that we are done at that point. Please give your answer in the following format: \\ \tt '{"action":"<action>", "description": "<room description>", "graph": <graph>}'.\\ \tt <action> needs to be in the format "GO: <direction>" where <direction> is one of [north, east, south, west]. Alternatively, choose "DONE" as your action once you have explored the entire map.\\ \tt <room description> should be a single sentence describing the room shown to you.\\ \tt <graph> represents the map in this format: {"nodes":[], "edges":{"north": [], "south": [], "east": [], "west": []}}\\ \tt Omit any other text and answer only following the format, not adding anything except the dictionary!\\ \tt \\ \tt Here is an example: \\ \tt We are in this room. From here we can go: south, west. What is your next instruction?\\ \tt {"action":"GO: north", "description": "We are in a kitchen with a red fridge.", "graph":{"nodes":["Kitchen"], "edges":{"north": [], "south": [], "east": [], "west": []}}}\\ \tt We have made a step and are now in this room. From here we can go: east. What is your next instruction?\\ \tt {"action":"GO: east", "description": "We are in a living room with a couch and a tv.", "graph":{"nodes":["Kitchen", "Living Room"], "edges":{"north": [["Kitchen", "Living Room"]], "south": [], "east": [], "west": []}}}\\ \tt You have made a step and are now in this room. From here we can go: west, south. What is your next instruction?\\ \tt {"action":"GO: south", "description": "We are in a bedroom with two beds and a nightstand.",  "graph":{"nodes":["Kitchen", "Living Room", "Bedroom"], "edges":{"north": [["Kitchen", "Living Room"]], "south": [], "east": [["Living Room", "Bedroom"]], "west": []}}}\\ \tt ...\\ \tt You have made a step and are now in this room. From here we can go: north. What is your next instruction?\\ \tt Example answer:\\ \tt {"action":"DONE", "description": "We are in a stairwell, the stair is curved.", "graph":"{...}"}\\ \tt \\ \tt Let us start.\\ \tt Currently available directions: west, north. What is your next instruction?\\ \tt  
	  } 
	   } 
	   } 
	 & & \\ 
 

    \theutterance \stepcounter{utterance}  

    & & \multicolumn{4}{p{0.6\linewidth}}{\cellcolor[rgb]{0.9,0.9,0.9}{%
	\makecell[{{p{\linewidth}}}]{% 
	  \tt {\tiny [GM$\rangle$B]}  
	 We are currently in this room. Please help me with the following task. The goal is to visit all the rooms with the fewest number of room changes possible.  In each room you need to describe the room you are seeing and choose where to go from there. Additionally, you need to provide a graph representing the map you have uncovered. Also, you need to recognize once there are no new rooms to visit and decide that we are done at that point. Please give your answer in the following format: \\ \tt '{"action":"<action>", "description": "<room description>", "graph": <graph>}'.\\ \tt <action> needs to be in the format "GO: <direction>" where <direction> is one of [north, east, south, west]. Alternatively, choose "DONE" as your action once you have explored the entire map.\\ \tt <room description> should be a single sentence describing the room shown to you.\\ \tt <graph> represents the map in this format: {"nodes":[], "edges":{"north": [], "south": [], "east": [], "west": []}}\\ \tt Omit any other text and answer only following the format, not adding anything except the dictionary!\\ \tt \\ \tt Here is an example: \\ \tt We are in this room. From here we can go: south, west. What is your next instruction?\\ \tt {"action":"GO: north", "description": "We are in a kitchen with a red fridge.", "graph":{"nodes":["Kitchen"], "edges":{"north": [], "south": [], "east": [], "west": []}}}\\ \tt We have made a step and are now in this room. From here we can go: east. What is your next instruction?\\ \tt {"action":"GO: east", "description": "We are in a living room with a couch and a tv.", "graph":{"nodes":["Kitchen", "Living Room"], "edges":{"north": [["Kitchen", "Living Room"]], "south": [], "east": [], "west": []}}}\\ \tt You have made a step and are now in this room. From here we can go: west, south. What is your next instruction?\\ \tt {"action":"GO: south", "description": "We are in a bedroom with two beds and a nightstand.",  "graph":{"nodes":["Kitchen", "Living Room", "Bedroom"], "edges":{"north": [["Kitchen", "Living Room"]], "south": [], "east": [["Living Room", "Bedroom"]], "west": []}}}\\ \tt ...\\ \tt You have made a step and are now in this room. From here we can go: north. What is your next instruction?\\ \tt Example answer:\\ \tt {"action":"DONE", "description": "We are in a stairwell, the stair is curved.", "graph":"{...}"}\\ \tt \\ \tt Let us start.\\ \tt Currently available directions: west, north. What is your next instruction?\\ \tt  
	  } 
	   } 
	   } 
	 & \\ 
 

    \theutterance \stepcounter{utterance}  

    & & & \multicolumn{4}{p{0.6\linewidth}}{\cellcolor[rgb]{1,0.85,0.72}{%
	\makecell[{{p{\linewidth}}}]{% 
	  \tt {\tiny [GM$\langle$B]}  
	 Based on the provided image and the context of the task, the room appears to be a compact, multi-functional space, likely designed for efficiency in a setting such as a hostel, dormitory, or a compact living arrangement. The presence of the ladder suggests that the sleeping area is above the sink and storage area, maximizing the use of vertical space in the room.\\ \tt \\ \tt Given the current directions available, "west" and "north," and considering the layout of the room, the next instruction would be to go "north." This would lead to exploring the area above the sink and storage, which might contain the sleeping area or additional storage space.\\ \tt \\ \tt Therefore, the next instruction is:\\ \tt ```json\\ \tt {\\ \tt   "action": "GO: north",\\ \tt   "description": "We are in a compact room with a ladder leading to a sleeping area, above a sink and storage unit."\\ \tt   "graph": {\\ \tt     "nodes": [],\\ \tt     "edges": {\\ \tt       "north": [],\\ \tt       "south": [],\\ \tt       "east": [],\\ \tt       "west": []\\ \tt     }\\ \tt   }\\ \tt }\\ \tt ```\\ \tt This instruction follows the format provided in the example, with the action being "GO: north," the description of the room, and the current state of the graph representing the map. 
	  } 
	   } 
	   } 
	  \\ 
 

    \theutterance \stepcounter{utterance}  

    & & & \multicolumn{2}{p{0.3\linewidth}}{\cellcolor[rgb]{0.95,0.95,0.95}{%
	\makecell[{{p{\linewidth}}}]{% 
	  \tt {\tiny [GM$|$GM]}  
	 Game aborted. 
	  } 
	   } 
	   } 
	 & & \\ 
 

    \theutterance \stepcounter{utterance}  

    & & & \multicolumn{2}{p{0.3\linewidth}}{\cellcolor[rgb]{0.95,0.95,0.95}{%
	\makecell[{{p{\linewidth}}}]{% 
	  \tt {\tiny [GM$|$GM]}  
	 True 
	  } 
	   } 
	   } 
	 & & \\ 
 

\end{supertabular}
}

\end{document}
