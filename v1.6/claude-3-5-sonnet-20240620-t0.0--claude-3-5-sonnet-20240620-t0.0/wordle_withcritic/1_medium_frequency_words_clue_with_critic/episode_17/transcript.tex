
\documentclass{article}
\usepackage{colortbl}
\usepackage{makecell}
\usepackage{multirow}
\usepackage{supertabular}

\begin{document}

\newcounter{utterance}

\twocolumn

{ \footnotesize  \setcounter{utterance}{1}
\setlength{\tabcolsep}{0pt}
\begin{supertabular}{c@{$\;$}|p{.15\linewidth}@{}p{.15\linewidth}p{.15\linewidth}p{.15\linewidth}p{.15\linewidth}p{.15\linewidth}}

    \# & $\;$A & \multicolumn{4}{c}{Game Master} & $\;\:$B\\
    \hline 

    \theutterance \stepcounter{utterance}  

    & & \multicolumn{4}{p{0.6\linewidth}}{\cellcolor[rgb]{0.9,0.9,0.9}{%
	\makecell[{{p{\linewidth}}}]{% 
	  \tt {\tiny [A$\langle$GM]}  
	 You are a language wizard who likes to guess words by using the given rules.\\ \tt \\ \tt Welcome to Wordle! You have six attempts to guess the target word, a valid English word of five lowercase letters (a-z). Please use the tags "guess:" and "explanation:" to provide a concise explanation for each guess.\\ \tt \\ \tt To help you make an informed guess, you will receive a clue for the word, such as\\ \tt clue: "snowy white"\\ \tt \\ \tt Here is an example guess based on the clue:\\ \tt guess: apple\\ \tt explanation: In the fairy tail Snow White, the girl is killed because she eats a poisoned apple. And the word apple has 5 letters.\\ \tt \\ \tt I will then indicate whether I agree or disagree with your guess and provide rationale, but agreeing with a guess does not confirm its correctness. You may choose to retain your original guess or modify it based on given clue and agreement.\\ \tt \\ \tt After each guess, your answer will be validated, and you will receive feedback indicating which letters are correct (green), which letters are correct but in the wrong position (yellow), and which letters are incorrect (red). This feedback can be useful in determining which letters to include or exclude in your next guess.\\ \tt \\ \tt For example, the feedback for "apple" might be:\\ \tt guess_feedback: a<yellow> p<yellow> p<green> l<yellow> e<red>\\ \tt \\ \tt Here are some rules to keep in mind:\\ \tt 1. Please do not generate any guess feedback or clue on your own\\ \tt 2. Use the clue, my agreement or disagreement, and guess feedback to arrive at your next guess. This should be reflected in your explanation as well.\\ \tt \\ \tt Let's begin with your first guess.\\ \tt \\ \tt clue:hungry\\ \tt  
	  } 
	   } 
	   } 
	 & \\ 
 

    \theutterance \stepcounter{utterance}  

    & \multicolumn{4}{p{0.6\linewidth}}{\cellcolor[rgb]{0.8,1,0.9}{%
	\makecell[{{p{\linewidth}}}]{% 
	  \tt {\tiny [A$\rangle$GM]}  
	 Understood. I'll use the given clue to make my first guess and provide an explanation.\\ \tt \\ \tt guess: feast\\ \tt explanation: The word "feast" is closely associated with the clue "hungry." A feast is a large meal typically prepared for a special occasion, and it's something that hungry people would eagerly anticipate. The word "feast" has five letters, making it a valid guess for Wordle. 
	  } 
	   } 
	   } 
	 & & \\ 
 

    \theutterance \stepcounter{utterance}  

    & & & \multicolumn{2}{p{0.3\linewidth}}{\cellcolor[rgb]{0.95,0.95,0.95}{%
	\makecell[{{p{\linewidth}}}]{% 
	  \tt {\tiny [GM$|$GM]}  
	 {'guess': 'feast', 'explanation': 'The word "feast" is closely associated with the clue "hungry." A feast is a large meal typically prepared for a special occasion, and it\'s something that hungry people would eagerly anticipate. The word "feast" has five letters, making it a valid guess for Wordle.'} 
	  } 
	   } 
	   } 
	 & & \\ 
 

    \theutterance \stepcounter{utterance}  

    & & \multicolumn{4}{p{0.6\linewidth}}{\cellcolor[rgb]{0.9,0.9,0.9}{%
	\makecell[{{p{\linewidth}}}]{% 
	  \tt {\tiny [GM$\rangle$B]}  
	 I need your assistance with a word game in which we need to find a 5-letter word using a clue, a guess and an explanation for the guess.\\ \tt \\ \tt For example, if the clue is "snow white" and the guess is "apple", you will get the information like this:\\ \tt clue:snow white\\ \tt guess:apple\\ \tt explanation:the fairy tale Snow White involves the girl eating a poisoned apple, which has 5 letters.\\ \tt \\ \tt Your task is to either agree or disagree with my guess based on the given clue.\\ \tt \\ \tt If you disagree, I will make another guess, and if you agree, I will proceed in the game using the same guess.\\ \tt \\ \tt Initially, guess feedback is not available, but as the game proceeds, it becomes available. The guess feedback is not for current guess, but rather an indication of what letters can be present in the current guess. A letter highlighted in green indicates that it is a correct letter in the correct position, while yellow indicates that it is a correct letter in the wrong position, and red indicates a wrong letter. At this point, you should use all the provided information, including the clue, guess, explanation, and guess feedback, to decide whether the given guess matches the clue, and also stick to guess feedback or not.\\ \tt \\ \tt Please respond in lowercase letters and stick to this format:\\ \tt agreement:yes or no\\ \tt explanation:your reason for doing so\\ \tt \\ \tt Please note that you must not guess any word based on the riddle clue and stick to the given format while responding.\\ \tt \\ \tt Let's begin!\\ \tt \\ \tt clue:hungry\\ \tt guess:feast\\ \tt explanation:The word "feast" is closely associated with the clue "hungry." A feast is a large meal typically prepared for a special occasion, and it's something that hungry people would eagerly anticipate. The word "feast" has five letters, making it a valid guess for Wordle.\\ \tt  
	  } 
	   } 
	   } 
	 & \\ 
 

    \theutterance \stepcounter{utterance}  

    & & & \multicolumn{4}{p{0.6\linewidth}}{\cellcolor[rgb]{1,0.85,0.72}{%
	\makecell[{{p{\linewidth}}}]{% 
	  \tt {\tiny [GM$\langle$B]}  
	 agreement:yes\\ \tt explanation:the word "feast" is a logical guess for the clue "hungry" as it represents a large meal that would satisfy hunger. it's a 5-letter word that fits the game's requirements and has a clear connection to the given clue. 
	  } 
	   } 
	   } 
	  \\ 
 

    \theutterance \stepcounter{utterance}  

    & & & \multicolumn{2}{p{0.3\linewidth}}{\cellcolor[rgb]{0.95,0.95,0.95}{%
	\makecell[{{p{\linewidth}}}]{% 
	  \tt {\tiny [GM$|$GM]}  
	 {'agreement': 'yes', 'explanation': 'the word "feast" is a logical guess for the clue "hungry" as it represents a large meal that would satisfy hunger. it\'s a 5-letter word that fits the game\'s requirements and has a clear connection to the given clue.'} 
	  } 
	   } 
	   } 
	 & & \\ 
 

    \theutterance \stepcounter{utterance}  

    & & & \multicolumn{2}{p{0.3\linewidth}}{\cellcolor[rgb]{0.95,0.95,0.95}{%
	\makecell[{{p{\linewidth}}}]{% 
	  \tt {\tiny [GM$|$GM]}  
	 Critic agrees with the Guesser -- Sharing the critic's explanation with the guesser 
	  } 
	   } 
	   } 
	 & & \\ 
 

    \theutterance \stepcounter{utterance}  

    & & \multicolumn{4}{p{0.6\linewidth}}{\cellcolor[rgb]{0.9,0.9,0.9}{%
	\makecell[{{p{\linewidth}}}]{% 
	  \tt {\tiny [A$\langle$GM]}  
	 clue:hungry\\ \tt guess_agreement:yes\\ \tt agreement_explanation:the word "feast" is a logical guess for the clue "hungry" as it represents a large meal that would satisfy hunger. it's a 5-letter word that fits the game's requirements and has a clear connection to the given clue.\\ \tt  
	  } 
	   } 
	   } 
	 & \\ 
 

    \theutterance \stepcounter{utterance}  

    & \multicolumn{4}{p{0.6\linewidth}}{\cellcolor[rgb]{0.8,1,0.9}{%
	\makecell[{{p{\linewidth}}}]{% 
	  \tt {\tiny [A$\rangle$GM]}  
	 Thank you for agreeing with my guess. I'll maintain my choice of "feast" for this round.\\ \tt \\ \tt guess: feast\\ \tt explanation: As we agreed, "feast" is a logical choice for the clue "hungry." It represents a large, satisfying meal that would alleviate hunger. The word fits the 5-letter requirement and has a strong connection to the clue. I'm keeping this guess to see how it aligns with the target word. 
	  } 
	   } 
	   } 
	 & & \\ 
 

    \theutterance \stepcounter{utterance}  

    & & & \multicolumn{2}{p{0.3\linewidth}}{\cellcolor[rgb]{0.95,0.95,0.95}{%
	\makecell[{{p{\linewidth}}}]{% 
	  \tt {\tiny [GM$|$GM]}  
	 {'guess': 'feast', 'explanation': 'As we agreed, "feast" is a logical choice for the clue "hungry." It represents a large, satisfying meal that would alleviate hunger. The word fits the 5-letter requirement and has a strong connection to the clue. I\'m keeping this guess to see how it aligns with the target word.'} 
	  } 
	   } 
	   } 
	 & & \\ 
 

    \theutterance \stepcounter{utterance}  

    & & & \multicolumn{2}{p{0.3\linewidth}}{\cellcolor[rgb]{0.95,0.95,0.95}{%
	\makecell[{{p{\linewidth}}}]{% 
	  \tt {\tiny [GM$|$GM]}  
	 No change in player1's guess\\ \tt guess_before_critic_opinion: feast\\ \tt                                                 critic_agreement: yes\\ \tt guess_after_critic_opinion: feast\\ \tt                                                 Proceeding with guess validation 
	  } 
	   } 
	   } 
	 & & \\ 
 

    \theutterance \stepcounter{utterance}  

    & & & \multicolumn{2}{p{0.3\linewidth}}{\cellcolor[rgb]{0.95,0.95,0.95}{%
	\makecell[{{p{\linewidth}}}]{% 
	  \tt {\tiny [GM$|$GM]}  
	 attempts: 1\\ \tt target_word = unfed\\ \tt target_word_clue = hungry\\ \tt guess: feast\\ \tt guess_feedback: f<yellow> e<yellow> a<red> s<red> t<red> 
	  } 
	   } 
	   } 
	 & & \\ 
 

    \theutterance \stepcounter{utterance}  

    & & \multicolumn{4}{p{0.6\linewidth}}{\cellcolor[rgb]{0.9,0.9,0.9}{%
	\makecell[{{p{\linewidth}}}]{% 
	  \tt {\tiny [A$\langle$GM]}  
	 guess_feedback: f<yellow> e<yellow> a<red> s<red> t<red> 
	  } 
	   } 
	   } 
	 & \\ 
 

    \theutterance \stepcounter{utterance}  

    & \multicolumn{4}{p{0.6\linewidth}}{\cellcolor[rgb]{0.8,1,0.9}{%
	\makecell[{{p{\linewidth}}}]{% 
	  \tt {\tiny [A$\rangle$GM]}  
	 guess: hefty\\ \tt explanation: Based on the previous feedback, we know that 'f' and 'e' are in the word but in different positions. "Hefty" incorporates these letters in new positions. It also relates to the clue "hungry" as hefty can mean large or substantial, which could describe a meal that would satisfy hunger. The word "hefty" maintains the 5-letter requirement and introduces new letters to test. 
	  } 
	   } 
	   } 
	 & & \\ 
 

    \theutterance \stepcounter{utterance}  

    & & & \multicolumn{2}{p{0.3\linewidth}}{\cellcolor[rgb]{0.95,0.95,0.95}{%
	\makecell[{{p{\linewidth}}}]{% 
	  \tt {\tiny [GM$|$GM]}  
	 {'guess': 'hefty', 'explanation': 'Based on the previous feedback, we know that \'f\' and \'e\' are in the word but in different positions. "Hefty" incorporates these letters in new positions. It also relates to the clue "hungry" as hefty can mean large or substantial, which could describe a meal that would satisfy hunger. The word "hefty" maintains the 5-letter requirement and introduces new letters to test.'} 
	  } 
	   } 
	   } 
	 & & \\ 
 

    \theutterance \stepcounter{utterance}  

    & & \multicolumn{4}{p{0.6\linewidth}}{\cellcolor[rgb]{0.9,0.9,0.9}{%
	\makecell[{{p{\linewidth}}}]{% 
	  \tt {\tiny [GM$\rangle$B]}  
	 clue:hungry\\ \tt guess:hefty\\ \tt explanation:Based on the previous feedback, we know that 'f' and 'e' are in the word but in different positions. "Hefty" incorporates these letters in new positions. It also relates to the clue "hungry" as hefty can mean large or substantial, which could describe a meal that would satisfy hunger. The word "hefty" maintains the 5-letter requirement and introduces new letters to test.\\ \tt guess_feedback: f<yellow> e<yellow> a<red> s<red> t<red> 
	  } 
	   } 
	   } 
	 & \\ 
 

    \theutterance \stepcounter{utterance}  

    & & & \multicolumn{4}{p{0.6\linewidth}}{\cellcolor[rgb]{1,0.85,0.72}{%
	\makecell[{{p{\linewidth}}}]{% 
	  \tt {\tiny [GM$\langle$B]}  
	 agreement: no\\ \tt explanation: While "hefty" does incorporate the yellow letters 'f' and 'e' from the previous feedback, which is good, it doesn't align well with the clue "hungry". The word "hefty" typically means heavy, substantial, or large, but it's not directly related to hunger or eating. Additionally, the feedback shows that 'a', 's', and 't' are not in the word at all, so including  
	  } 
	   } 
	   } 
	  \\ 
 

    \theutterance \stepcounter{utterance}  

    & & & \multicolumn{2}{p{0.3\linewidth}}{\cellcolor[rgb]{0.95,0.95,0.95}{%
	\makecell[{{p{\linewidth}}}]{% 
	  \tt {\tiny [GM$|$GM]}  
	 {'agreement': 'no', 'explanation': 'While "hefty" does incorporate the yellow letters \'f\' and \'e\' from the previous feedback, which is good, it doesn\'t align well with the clue "hungry". The word "hefty" typically means heavy, substantial, or large, but it\'s not directly related to hunger or eating. Additionally, the feedback shows that \'a\', \'s\', and \'t\' are not in the word at all, so including'} 
	  } 
	   } 
	   } 
	 & & \\ 
 

    \theutterance \stepcounter{utterance}  

    & & & \multicolumn{2}{p{0.3\linewidth}}{\cellcolor[rgb]{0.95,0.95,0.95}{%
	\makecell[{{p{\linewidth}}}]{% 
	  \tt {\tiny [GM$|$GM]}  
	 Critic disagrees with the Guesser -- Sharing the critic's explanation with the guesser 
	  } 
	   } 
	   } 
	 & & \\ 
 

    \theutterance \stepcounter{utterance}  

    & & \multicolumn{4}{p{0.6\linewidth}}{\cellcolor[rgb]{0.9,0.9,0.9}{%
	\makecell[{{p{\linewidth}}}]{% 
	  \tt {\tiny [A$\langle$GM]}  
	 clue:hungry\\ \tt guess_agreement:no\\ \tt agreement_explanation:While "hefty" does incorporate the yellow letters 'f' and 'e' from the previous feedback, which is good, it doesn't align well with the clue "hungry". The word "hefty" typically means heavy, substantial, or large, but it's not directly related to hunger or eating. Additionally, the feedback shows that 'a', 's', and 't' are not in the word at all, so including\\ \tt  
	  } 
	   } 
	   } 
	 & \\ 
 

    \theutterance \stepcounter{utterance}  

    & \multicolumn{4}{p{0.6\linewidth}}{\cellcolor[rgb]{0.8,1,0.9}{%
	\makecell[{{p{\linewidth}}}]{% 
	  \tt {\tiny [A$\rangle$GM]}  
	 I apologize for the oversight in my previous guess. You're absolutely right, and I appreciate your feedback. Let me make a more appropriate guess based on the clue and the feedback we've received so far.\\ \tt \\ \tt guess: fries\\ \tt explanation: This guess aligns better with the clue "hungry" as fries are a popular food that people often crave when hungry. It retains the 'f' from the previous feedback but places it at the beginning, and keeps  
	  } 
	   } 
	   } 
	 & & \\ 
 

    \theutterance \stepcounter{utterance}  

    & & & \multicolumn{2}{p{0.3\linewidth}}{\cellcolor[rgb]{0.95,0.95,0.95}{%
	\makecell[{{p{\linewidth}}}]{% 
	  \tt {\tiny [GM$|$GM]}  
	 {'guess': 'fries', 'explanation': 'This guess aligns better with the clue "hungry" as fries are a popular food that people often crave when hungry. It retains the \'f\' from the previous feedback but places it at the beginning, and keeps'} 
	  } 
	   } 
	   } 
	 & & \\ 
 

    \theutterance \stepcounter{utterance}  

    & & & \multicolumn{2}{p{0.3\linewidth}}{\cellcolor[rgb]{0.95,0.95,0.95}{%
	\makecell[{{p{\linewidth}}}]{% 
	  \tt {\tiny [GM$|$GM]}  
	 Change in player1's guess\\ \tt guess_before_critic_opinion: hefty\\ \tt                                                 critic_agreement: no\\ \tt guess_after_critic_opinion: fries\\ \tt                                                 Proceeding with guess validation 
	  } 
	   } 
	   } 
	 & & \\ 
 

    \theutterance \stepcounter{utterance}  

    & & & \multicolumn{2}{p{0.3\linewidth}}{\cellcolor[rgb]{0.95,0.95,0.95}{%
	\makecell[{{p{\linewidth}}}]{% 
	  \tt {\tiny [GM$|$GM]}  
	 attempts: 2\\ \tt target_word = unfed\\ \tt target_word_clue = hungry\\ \tt guess: fries\\ \tt guess_feedback: f<yellow> r<red> i<red> e<green> s<red> 
	  } 
	   } 
	   } 
	 & & \\ 
 

    \theutterance \stepcounter{utterance}  

    & & \multicolumn{4}{p{0.6\linewidth}}{\cellcolor[rgb]{0.9,0.9,0.9}{%
	\makecell[{{p{\linewidth}}}]{% 
	  \tt {\tiny [A$\langle$GM]}  
	 guess_feedback: f<yellow> r<red> i<red> e<green> s<red> 
	  } 
	   } 
	   } 
	 & \\ 
 

    \theutterance \stepcounter{utterance}  

    & \multicolumn{4}{p{0.6\linewidth}}{\cellcolor[rgb]{0.8,1,0.9}{%
	\makecell[{{p{\linewidth}}}]{% 
	  \tt {\tiny [A$\rangle$GM]}  
	 guess:fleck\\ \tt explanation:Based on the previous feedback, we know that 'f' is in the word but not in the first position, and 'e' is correctly placed in the fourth position. "Fleck" incorporates these letters accordingly. While "fleck" doesn't directly relate to the clue "hungry," we're prioritizing the letter placement information we've gathered. The word "fleck" introduces new letters 'l', 'c', 
	  } 
	   } 
	   } 
	 & & \\ 
 

    \theutterance \stepcounter{utterance}  

    & & & \multicolumn{2}{p{0.3\linewidth}}{\cellcolor[rgb]{0.95,0.95,0.95}{%
	\makecell[{{p{\linewidth}}}]{% 
	  \tt {\tiny [GM$|$GM]}  
	 {'guess': 'fleck', 'explanation': 'Based on the previous feedback, we know that \'f\' is in the word but not in the first position, and \'e\' is correctly placed in the fourth position. "Fleck" incorporates these letters accordingly. While "fleck" doesn\'t directly relate to the clue "hungry," we\'re prioritizing the letter placement information we\'ve gathered. The word "fleck" introduces new letters \'l\', \'c\','} 
	  } 
	   } 
	   } 
	 & & \\ 
 

    \theutterance \stepcounter{utterance}  

    & & \multicolumn{4}{p{0.6\linewidth}}{\cellcolor[rgb]{0.9,0.9,0.9}{%
	\makecell[{{p{\linewidth}}}]{% 
	  \tt {\tiny [GM$\rangle$B]}  
	 clue:hungry\\ \tt guess:fleck\\ \tt explanation:Based on the previous feedback, we know that 'f' is in the word but not in the first position, and 'e' is correctly placed in the fourth position. "Fleck" incorporates these letters accordingly. While "fleck" doesn't directly relate to the clue "hungry," we're prioritizing the letter placement information we've gathered. The word "fleck" introduces new letters 'l', 'c',\\ \tt guess_feedback: f<yellow> r<red> i<red> e<green> s<red> 
	  } 
	   } 
	   } 
	 & \\ 
 

    \theutterance \stepcounter{utterance}  

    & & & \multicolumn{4}{p{0.6\linewidth}}{\cellcolor[rgb]{1,0.85,0.72}{%
	\makecell[{{p{\linewidth}}}]{% 
	  \tt {\tiny [GM$\langle$B]}  
	 agreement: no\\ \tt explanation: While "fleck" does incorporate the correct placement of 'e' in the fourth position and includes 'f' in a different position as per the feedback, it doesn't align well with the clue "hungry". Additionally, the guess feedback shows that 'r' and 'i' are not in the word, but these letters aren't present in "fleck". The guess also doesn't account for the possibility of other correct letters from previous 
	  } 
	   } 
	   } 
	  \\ 
 

    \theutterance \stepcounter{utterance}  

    & & & \multicolumn{2}{p{0.3\linewidth}}{\cellcolor[rgb]{0.95,0.95,0.95}{%
	\makecell[{{p{\linewidth}}}]{% 
	  \tt {\tiny [GM$|$GM]}  
	 {'agreement': 'no', 'explanation': 'While "fleck" does incorporate the correct placement of \'e\' in the fourth position and includes \'f\' in a different position as per the feedback, it doesn\'t align well with the clue "hungry". Additionally, the guess feedback shows that \'r\' and \'i\' are not in the word, but these letters aren\'t present in "fleck". The guess also doesn\'t account for the possibility of other correct letters from previous'} 
	  } 
	   } 
	   } 
	 & & \\ 
 

    \theutterance \stepcounter{utterance}  

    & & & \multicolumn{2}{p{0.3\linewidth}}{\cellcolor[rgb]{0.95,0.95,0.95}{%
	\makecell[{{p{\linewidth}}}]{% 
	  \tt {\tiny [GM$|$GM]}  
	 Critic disagrees with the Guesser -- Sharing the critic's explanation with the guesser 
	  } 
	   } 
	   } 
	 & & \\ 
 

    \theutterance \stepcounter{utterance}  

    & & \multicolumn{4}{p{0.6\linewidth}}{\cellcolor[rgb]{0.9,0.9,0.9}{%
	\makecell[{{p{\linewidth}}}]{% 
	  \tt {\tiny [A$\langle$GM]}  
	 clue:hungry\\ \tt guess_agreement:no\\ \tt agreement_explanation:While "fleck" does incorporate the correct placement of 'e' in the fourth position and includes 'f' in a different position as per the feedback, it doesn't align well with the clue "hungry". Additionally, the guess feedback shows that 'r' and 'i' are not in the word, but these letters aren't present in "fleck". The guess also doesn't account for the possibility of other correct letters from previous\\ \tt  
	  } 
	   } 
	   } 
	 & \\ 
 

    \theutterance \stepcounter{utterance}  

    & \multicolumn{4}{p{0.6\linewidth}}{\cellcolor[rgb]{0.8,1,0.9}{%
	\makecell[{{p{\linewidth}}}]{% 
	  \tt {\tiny [A$\rangle$GM]}  
	 I apologize for the oversight. You're right, and I'll adjust my approach.\\ \tt \\ \tt guess:fiend\\ \tt explanation:This guess aligns better with both the clue and the feedback. "Fiend" relates to "hungry" as it can describe someone with an intense craving or appetite (e.g., "coffee fiend" or "food fiend"). It keeps 'f' (now in the first position) and 'e' in the fourth position 
	  } 
	   } 
	   } 
	 & & \\ 
 

    \theutterance \stepcounter{utterance}  

    & & & \multicolumn{2}{p{0.3\linewidth}}{\cellcolor[rgb]{0.95,0.95,0.95}{%
	\makecell[{{p{\linewidth}}}]{% 
	  \tt {\tiny [GM$|$GM]}  
	 {'guess': 'fiend', 'explanation': 'This guess aligns better with both the clue and the feedback. "Fiend" relates to "hungry" as it can describe someone with an intense craving or appetite (e.g., "coffee fiend" or "food fiend"). It keeps \'f\' (now in the first position) and \'e\' in the fourth position'} 
	  } 
	   } 
	   } 
	 & & \\ 
 

    \theutterance \stepcounter{utterance}  

    & & & \multicolumn{2}{p{0.3\linewidth}}{\cellcolor[rgb]{0.95,0.95,0.95}{%
	\makecell[{{p{\linewidth}}}]{% 
	  \tt {\tiny [GM$|$GM]}  
	 Change in player1's guess\\ \tt guess_before_critic_opinion: fleck\\ \tt                                                 critic_agreement: no\\ \tt guess_after_critic_opinion: fiend\\ \tt                                                 Proceeding with guess validation 
	  } 
	   } 
	   } 
	 & & \\ 
 

    \theutterance \stepcounter{utterance}  

    & & & \multicolumn{2}{p{0.3\linewidth}}{\cellcolor[rgb]{0.95,0.95,0.95}{%
	\makecell[{{p{\linewidth}}}]{% 
	  \tt {\tiny [GM$|$GM]}  
	 attempts: 3\\ \tt target_word = unfed\\ \tt target_word_clue = hungry\\ \tt guess: fiend\\ \tt guess_feedback: f<yellow> i<red> e<yellow> n<yellow> d<green> 
	  } 
	   } 
	   } 
	 & & \\ 
 

    \theutterance \stepcounter{utterance}  

    & & \multicolumn{4}{p{0.6\linewidth}}{\cellcolor[rgb]{0.9,0.9,0.9}{%
	\makecell[{{p{\linewidth}}}]{% 
	  \tt {\tiny [A$\langle$GM]}  
	 guess_feedback: f<yellow> i<red> e<yellow> n<yellow> d<green> 
	  } 
	   } 
	   } 
	 & \\ 
 

    \theutterance \stepcounter{utterance}  

    & \multicolumn{4}{p{0.6\linewidth}}{\cellcolor[rgb]{0.8,1,0.9}{%
	\makecell[{{p{\linewidth}}}]{% 
	  \tt {\tiny [A$\rangle$GM]}  
	 guess:blend\\ \tt explanation:Based on the latest feedback, we know that 'd' is correct in the last position, 'e' and 'n' are in the word but in different positions, and 'f' is still in the word but not in the first position. "Blend" incorporates these letters accordingly. While "blend" doesn't directly relate to the clue "hungry," we're prioritizing the letter placement information we've gathered. The word " 
	  } 
	   } 
	   } 
	 & & \\ 
 

    \theutterance \stepcounter{utterance}  

    & & & \multicolumn{2}{p{0.3\linewidth}}{\cellcolor[rgb]{0.95,0.95,0.95}{%
	\makecell[{{p{\linewidth}}}]{% 
	  \tt {\tiny [GM$|$GM]}  
	 {'guess': 'blend', 'explanation': 'Based on the latest feedback, we know that \'d\' is correct in the last position, \'e\' and \'n\' are in the word but in different positions, and \'f\' is still in the word but not in the first position. "Blend" incorporates these letters accordingly. While "blend" doesn\'t directly relate to the clue "hungry," we\'re prioritizing the letter placement information we\'ve gathered. The word "'} 
	  } 
	   } 
	   } 
	 & & \\ 
 

    \theutterance \stepcounter{utterance}  

    & & \multicolumn{4}{p{0.6\linewidth}}{\cellcolor[rgb]{0.9,0.9,0.9}{%
	\makecell[{{p{\linewidth}}}]{% 
	  \tt {\tiny [GM$\rangle$B]}  
	 clue:hungry\\ \tt guess:blend\\ \tt explanation:Based on the latest feedback, we know that 'd' is correct in the last position, 'e' and 'n' are in the word but in different positions, and 'f' is still in the word but not in the first position. "Blend" incorporates these letters accordingly. While "blend" doesn't directly relate to the clue "hungry," we're prioritizing the letter placement information we've gathered. The word "\\ \tt guess_feedback: f<yellow> i<red> e<yellow> n<yellow> d<green> 
	  } 
	   } 
	   } 
	 & \\ 
 

    \theutterance \stepcounter{utterance}  

    & & & \multicolumn{4}{p{0.6\linewidth}}{\cellcolor[rgb]{1,0.85,0.72}{%
	\makecell[{{p{\linewidth}}}]{% 
	  \tt {\tiny [GM$\langle$B]}  
	 agreement: no\\ \tt explanation: While "blend" does incorporate some of the correct letters from the previous feedback ('e', 'n', and 'd'), it doesn't use the confirmed 'f' that we know is in the word. Additionally, the guess feedback shows that 'b' and 'l' are not in the word, as they are not highlighted. The word we're looking for should have 'f', 'e', 'n', and  
	  } 
	   } 
	   } 
	  \\ 
 

    \theutterance \stepcounter{utterance}  

    & & & \multicolumn{2}{p{0.3\linewidth}}{\cellcolor[rgb]{0.95,0.95,0.95}{%
	\makecell[{{p{\linewidth}}}]{% 
	  \tt {\tiny [GM$|$GM]}  
	 {'agreement': 'no', 'explanation': 'While "blend" does incorporate some of the correct letters from the previous feedback (\'e\', \'n\', and \'d\'), it doesn\'t use the confirmed \'f\' that we know is in the word. Additionally, the guess feedback shows that \'b\' and \'l\' are not in the word, as they are not highlighted. The word we\'re looking for should have \'f\', \'e\', \'n\', and'} 
	  } 
	   } 
	   } 
	 & & \\ 
 

    \theutterance \stepcounter{utterance}  

    & & & \multicolumn{2}{p{0.3\linewidth}}{\cellcolor[rgb]{0.95,0.95,0.95}{%
	\makecell[{{p{\linewidth}}}]{% 
	  \tt {\tiny [GM$|$GM]}  
	 Critic disagrees with the Guesser -- Sharing the critic's explanation with the guesser 
	  } 
	   } 
	   } 
	 & & \\ 
 

    \theutterance \stepcounter{utterance}  

    & & \multicolumn{4}{p{0.6\linewidth}}{\cellcolor[rgb]{0.9,0.9,0.9}{%
	\makecell[{{p{\linewidth}}}]{% 
	  \tt {\tiny [A$\langle$GM]}  
	 clue:hungry\\ \tt guess_agreement:no\\ \tt agreement_explanation:While "blend" does incorporate some of the correct letters from the previous feedback ('e', 'n', and 'd'), it doesn't use the confirmed 'f' that we know is in the word. Additionally, the guess feedback shows that 'b' and 'l' are not in the word, as they are not highlighted. The word we're looking for should have 'f', 'e', 'n', and\\ \tt  
	  } 
	   } 
	   } 
	 & \\ 
 

    \theutterance \stepcounter{utterance}  

    & \multicolumn{4}{p{0.6\linewidth}}{\cellcolor[rgb]{0.8,1,0.9}{%
	\makecell[{{p{\linewidth}}}]{% 
	  \tt {\tiny [A$\rangle$GM]}  
	 I apologize for the oversight. You're absolutely right, and I'll make a more informed guess based on all the information we have.\\ \tt \\ \tt guess:fined\\ \tt explanation:This guess aligns better with both the clue and all the feedback we've received. It includes 'f' (not in the first position), 'e' (now in the second position), 'n' (in the third position), and 'd' (correctly placed in the last position). 
	  } 
	   } 
	   } 
	 & & \\ 
 

    \theutterance \stepcounter{utterance}  

    & & & \multicolumn{2}{p{0.3\linewidth}}{\cellcolor[rgb]{0.95,0.95,0.95}{%
	\makecell[{{p{\linewidth}}}]{% 
	  \tt {\tiny [GM$|$GM]}  
	 {'guess': 'fined', 'explanation': "This guess aligns better with both the clue and all the feedback we've received. It includes 'f' (not in the first position), 'e' (now in the second position), 'n' (in the third position), and 'd' (correctly placed in the last position)."} 
	  } 
	   } 
	   } 
	 & & \\ 
 

    \theutterance \stepcounter{utterance}  

    & & & \multicolumn{2}{p{0.3\linewidth}}{\cellcolor[rgb]{0.95,0.95,0.95}{%
	\makecell[{{p{\linewidth}}}]{% 
	  \tt {\tiny [GM$|$GM]}  
	 Change in player1's guess\\ \tt guess_before_critic_opinion: blend\\ \tt                                                 critic_agreement: no\\ \tt guess_after_critic_opinion: fined\\ \tt                                                 Proceeding with guess validation 
	  } 
	   } 
	   } 
	 & & \\ 
 

    \theutterance \stepcounter{utterance}  

    & & & \multicolumn{2}{p{0.3\linewidth}}{\cellcolor[rgb]{0.95,0.95,0.95}{%
	\makecell[{{p{\linewidth}}}]{% 
	  \tt {\tiny [GM$|$GM]}  
	 attempts: 4\\ \tt target_word = unfed\\ \tt target_word_clue = hungry\\ \tt guess: fined\\ \tt guess_feedback: f<yellow> i<red> n<yellow> e<green> d<green> 
	  } 
	   } 
	   } 
	 & & \\ 
 

    \theutterance \stepcounter{utterance}  

    & & \multicolumn{4}{p{0.6\linewidth}}{\cellcolor[rgb]{0.9,0.9,0.9}{%
	\makecell[{{p{\linewidth}}}]{% 
	  \tt {\tiny [A$\langle$GM]}  
	 guess_feedback: f<yellow> i<red> n<yellow> e<green> d<green> 
	  } 
	   } 
	   } 
	 & \\ 
 

    \theutterance \stepcounter{utterance}  

    & \multicolumn{4}{p{0.6\linewidth}}{\cellcolor[rgb]{0.8,1,0.9}{%
	\makecell[{{p{\linewidth}}}]{% 
	  \tt {\tiny [A$\rangle$GM]}  
	 guess:fened\\ \tt explanation:Based on all the feedback we've received, we now know that 'e' is correct in the fourth position and 'd' is correct in the fifth position. We also know that 'f' and 'n' are in the word but not in their current positions. This guess, "fened," incorporates all of this information. While "fened" isn't a common word (it's an archaic past tense of " 
	  } 
	   } 
	   } 
	 & & \\ 
 

    \theutterance \stepcounter{utterance}  

    & & & \multicolumn{2}{p{0.3\linewidth}}{\cellcolor[rgb]{0.95,0.95,0.95}{%
	\makecell[{{p{\linewidth}}}]{% 
	  \tt {\tiny [GM$|$GM]}  
	 {'guess': 'fened', 'explanation': 'Based on all the feedback we\'ve received, we now know that \'e\' is correct in the fourth position and \'d\' is correct in the fifth position. We also know that \'f\' and \'n\' are in the word but not in their current positions. This guess, "fened," incorporates all of this information. While "fened" isn\'t a common word (it\'s an archaic past tense of "'} 
	  } 
	   } 
	   } 
	 & & \\ 
 

    \theutterance \stepcounter{utterance}  

    & & & \multicolumn{2}{p{0.3\linewidth}}{\cellcolor[rgb]{0.95,0.95,0.95}{%
	\makecell[{{p{\linewidth}}}]{% 
	  \tt {\tiny [GM$|$GM]}  
	 Guesser Error: NOT_VALID_ENGLISH_WORD while parsing Player 1's (model: claude-3-5-sonnet-20240620) response 
	  } 
	   } 
	   } 
	 & & \\ 
 

    \theutterance \stepcounter{utterance}  

    & & \multicolumn{4}{p{0.6\linewidth}}{\cellcolor[rgb]{0.9,0.9,0.9}{%
	\makecell[{{p{\linewidth}}}]{% 
	  \tt {\tiny [A$\langle$GM]}  
	 your guess is not a valid word for this game. Please try again 
	  } 
	   } 
	   } 
	 & \\ 
 

    \theutterance \stepcounter{utterance}  

    & \multicolumn{4}{p{0.6\linewidth}}{\cellcolor[rgb]{0.8,1,0.9}{%
	\makecell[{{p{\linewidth}}}]{% 
	  \tt {\tiny [A$\rangle$GM]}  
	 I apologize for the invalid guess. Let me try again with a valid word.\\ \tt \\ \tt guess:fends\\ \tt explanation:This guess incorporates all the correct information we've gathered so far. 'F' is included but not in the first position, 'e' is in the correct fourth position, 'n' is included but moved to the third position, and 'd' remains in the correct fifth position. "Fends" is a valid English word meaning to war 
	  } 
	   } 
	   } 
	 & & \\ 
 

    \theutterance \stepcounter{utterance}  

    & & & \multicolumn{2}{p{0.3\linewidth}}{\cellcolor[rgb]{0.95,0.95,0.95}{%
	\makecell[{{p{\linewidth}}}]{% 
	  \tt {\tiny [GM$|$GM]}  
	 {'guess': 'fends', 'explanation': 'This guess incorporates all the correct information we\'ve gathered so far. \'F\' is included but not in the first position, \'e\' is in the correct fourth position, \'n\' is included but moved to the third position, and \'d\' remains in the correct fifth position. "Fends" is a valid English word meaning to war'} 
	  } 
	   } 
	   } 
	 & & \\ 
 

    \theutterance \stepcounter{utterance}  

    & & \multicolumn{4}{p{0.6\linewidth}}{\cellcolor[rgb]{0.9,0.9,0.9}{%
	\makecell[{{p{\linewidth}}}]{% 
	  \tt {\tiny [GM$\rangle$B]}  
	 clue:hungry\\ \tt guess:fends\\ \tt explanation:This guess incorporates all the correct information we've gathered so far. 'F' is included but not in the first position, 'e' is in the correct fourth position, 'n' is included but moved to the third position, and 'd' remains in the correct fifth position. "Fends" is a valid English word meaning to war\\ \tt guess_feedback: f<yellow> i<red> n<yellow> e<green> d<green> 
	  } 
	   } 
	   } 
	 & \\ 
 

    \theutterance \stepcounter{utterance}  

    & & & \multicolumn{4}{p{0.6\linewidth}}{\cellcolor[rgb]{1,0.85,0.72}{%
	\makecell[{{p{\linewidth}}}]{% 
	  \tt {\tiny [GM$\langle$B]}  
	 agreement: no\\ \tt explanation: While "fends" does incorporate the correct letter placements based on the previous feedback, it doesn't align well with the clue "hungry." The word "fends" means to ward off or defend against something, which doesn't have a clear connection to hunger or eating. Additionally, we know from the feedback that 'i' is not in the word, but "fends" doesn't introduce any new potentially correct letters to help us solve the puzzle 
	  } 
	   } 
	   } 
	  \\ 
 

    \theutterance \stepcounter{utterance}  

    & & & \multicolumn{2}{p{0.3\linewidth}}{\cellcolor[rgb]{0.95,0.95,0.95}{%
	\makecell[{{p{\linewidth}}}]{% 
	  \tt {\tiny [GM$|$GM]}  
	 {'agreement': 'no', 'explanation': 'While "fends" does incorporate the correct letter placements based on the previous feedback, it doesn\'t align well with the clue "hungry." The word "fends" means to ward off or defend against something, which doesn\'t have a clear connection to hunger or eating. Additionally, we know from the feedback that \'i\' is not in the word, but "fends" doesn\'t introduce any new potentially correct letters to help us solve the puzzle'} 
	  } 
	   } 
	   } 
	 & & \\ 
 

    \theutterance \stepcounter{utterance}  

    & & & \multicolumn{2}{p{0.3\linewidth}}{\cellcolor[rgb]{0.95,0.95,0.95}{%
	\makecell[{{p{\linewidth}}}]{% 
	  \tt {\tiny [GM$|$GM]}  
	 Critic disagrees with the Guesser -- Sharing the critic's explanation with the guesser 
	  } 
	   } 
	   } 
	 & & \\ 
 

    \theutterance \stepcounter{utterance}  

    & & \multicolumn{4}{p{0.6\linewidth}}{\cellcolor[rgb]{0.9,0.9,0.9}{%
	\makecell[{{p{\linewidth}}}]{% 
	  \tt {\tiny [A$\langle$GM]}  
	 clue:hungry\\ \tt guess_agreement:no\\ \tt agreement_explanation:While "fends" does incorporate the correct letter placements based on the previous feedback, it doesn't align well with the clue "hungry." The word "fends" means to ward off or defend against something, which doesn't have a clear connection to hunger or eating. Additionally, we know from the feedback that 'i' is not in the word, but "fends" doesn't introduce any new potentially correct letters to help us solve the puzzle\\ \tt  
	  } 
	   } 
	   } 
	 & \\ 
 

    \theutterance \stepcounter{utterance}  

    & \multicolumn{4}{p{0.6\linewidth}}{\cellcolor[rgb]{0.8,1,0.9}{%
	\makecell[{{p{\linewidth}}}]{% 
	  \tt {\tiny [A$\rangle$GM]}  
	 I apologize for the oversight. You're right, and I'll make a more appropriate guess.\\ \tt \\ \tt guess:feend\\ \tt explanation:This guess aligns better with both the clue and the feedback we've received. "Feend" is an archaic spelling of "fiend," which can relate to "hungry" as it describes someone with an intense craving or appetite (e.g., "food fiend"). It keeps 'f' (not in the first position 
	  } 
	   } 
	   } 
	 & & \\ 
 

    \theutterance \stepcounter{utterance}  

    & & & \multicolumn{2}{p{0.3\linewidth}}{\cellcolor[rgb]{0.95,0.95,0.95}{%
	\makecell[{{p{\linewidth}}}]{% 
	  \tt {\tiny [GM$|$GM]}  
	 {'guess': 'feend', 'explanation': 'This guess aligns better with both the clue and the feedback we\'ve received. "Feend" is an archaic spelling of "fiend," which can relate to "hungry" as it describes someone with an intense craving or appetite (e.g., "food fiend"). It keeps \'f\' (not in the first position'} 
	  } 
	   } 
	   } 
	 & & \\ 
 

    \theutterance \stepcounter{utterance}  

    & & & \multicolumn{2}{p{0.3\linewidth}}{\cellcolor[rgb]{0.95,0.95,0.95}{%
	\makecell[{{p{\linewidth}}}]{% 
	  \tt {\tiny [GM$|$GM]}  
	 Guesser Error: NOT_VALID_ENGLISH_WORD while parsing Player 1's (model: claude-3-5-sonnet-20240620) response 
	  } 
	   } 
	   } 
	 & & \\ 
 

    \theutterance \stepcounter{utterance}  

    & & \multicolumn{4}{p{0.6\linewidth}}{\cellcolor[rgb]{0.9,0.9,0.9}{%
	\makecell[{{p{\linewidth}}}]{% 
	  \tt {\tiny [A$\langle$GM]}  
	 your guess is not a valid word for this game. Please try again 
	  } 
	   } 
	   } 
	 & \\ 
 

    \theutterance \stepcounter{utterance}  

    & \multicolumn{4}{p{0.6\linewidth}}{\cellcolor[rgb]{0.8,1,0.9}{%
	\makecell[{{p{\linewidth}}}]{% 
	  \tt {\tiny [A$\rangle$GM]}  
	 I apologize for the invalid guess. Let me try again with a valid word.\\ \tt \\ \tt guess:feted\\ \tt explanation:This guess incorporates the correct letter placements we've gathered so far: 'f' is included but not in the first position, 'e' is in the correct fourth position, and 'd' remains in the correct fifth position. "Feted" is a valid English word meaning "honored or entertained lavishly," which can be loosely 
	  } 
	   } 
	   } 
	 & & \\ 
 

    \theutterance \stepcounter{utterance}  

    & & & \multicolumn{2}{p{0.3\linewidth}}{\cellcolor[rgb]{0.95,0.95,0.95}{%
	\makecell[{{p{\linewidth}}}]{% 
	  \tt {\tiny [GM$|$GM]}  
	 {'guess': 'feted', 'explanation': 'This guess incorporates the correct letter placements we\'ve gathered so far: \'f\' is included but not in the first position, \'e\' is in the correct fourth position, and \'d\' remains in the correct fifth position. "Feted" is a valid English word meaning "honored or entertained lavishly," which can be loosely'} 
	  } 
	   } 
	   } 
	 & & \\ 
 

    \theutterance \stepcounter{utterance}  

    & & & \multicolumn{2}{p{0.3\linewidth}}{\cellcolor[rgb]{0.95,0.95,0.95}{%
	\makecell[{{p{\linewidth}}}]{% 
	  \tt {\tiny [GM$|$GM]}  
	 Change in player1's guess\\ \tt guess_before_critic_opinion: fends\\ \tt                                                 critic_agreement: no\\ \tt guess_after_critic_opinion: feted\\ \tt                                                 Proceeding with guess validation 
	  } 
	   } 
	   } 
	 & & \\ 
 

    \theutterance \stepcounter{utterance}  

    & & & \multicolumn{2}{p{0.3\linewidth}}{\cellcolor[rgb]{0.95,0.95,0.95}{%
	\makecell[{{p{\linewidth}}}]{% 
	  \tt {\tiny [GM$|$GM]}  
	 attempts: 5\\ \tt target_word = unfed\\ \tt target_word_clue = hungry\\ \tt guess: feted\\ \tt guess_feedback: f<yellow> e<red> t<red> e<green> d<green> 
	  } 
	   } 
	   } 
	 & & \\ 
 

    \theutterance \stepcounter{utterance}  

    & & \multicolumn{4}{p{0.6\linewidth}}{\cellcolor[rgb]{0.9,0.9,0.9}{%
	\makecell[{{p{\linewidth}}}]{% 
	  \tt {\tiny [A$\langle$GM]}  
	 guess_feedback: f<yellow> e<red> t<red> e<green> d<green> 
	  } 
	   } 
	   } 
	 & \\ 
 

    \theutterance \stepcounter{utterance}  

    & \multicolumn{4}{p{0.6\linewidth}}{\cellcolor[rgb]{0.8,1,0.9}{%
	\makecell[{{p{\linewidth}}}]{% 
	  \tt {\tiny [A$\rangle$GM]}  
	 guess:found\\ \tt explanation:This guess incorporates the correct letter placements we've confirmed: 'e' in the fourth position and 'd' in the fifth position. It also includes 'f' in a new position (second) as we know it's in the word but not in the first position. "Found" relates to the clue "hungry" as it could be interpreted as having discovered food when hungry. Additionally, it introduces a new vowel 'o' 
	  } 
	   } 
	   } 
	 & & \\ 
 

    \theutterance \stepcounter{utterance}  

    & & & \multicolumn{2}{p{0.3\linewidth}}{\cellcolor[rgb]{0.95,0.95,0.95}{%
	\makecell[{{p{\linewidth}}}]{% 
	  \tt {\tiny [GM$|$GM]}  
	 {'guess': 'found', 'explanation': 'This guess incorporates the correct letter placements we\'ve confirmed: \'e\' in the fourth position and \'d\' in the fifth position. It also includes \'f\' in a new position (second) as we know it\'s in the word but not in the first position. "Found" relates to the clue "hungry" as it could be interpreted as having discovered food when hungry. Additionally, it introduces a new vowel \'o\''} 
	  } 
	   } 
	   } 
	 & & \\ 
 

    \theutterance \stepcounter{utterance}  

    & & \multicolumn{4}{p{0.6\linewidth}}{\cellcolor[rgb]{0.9,0.9,0.9}{%
	\makecell[{{p{\linewidth}}}]{% 
	  \tt {\tiny [GM$\rangle$B]}  
	 clue:hungry\\ \tt guess:found\\ \tt explanation:This guess incorporates the correct letter placements we've confirmed: 'e' in the fourth position and 'd' in the fifth position. It also includes 'f' in a new position (second) as we know it's in the word but not in the first position. "Found" relates to the clue "hungry" as it could be interpreted as having discovered food when hungry. Additionally, it introduces a new vowel 'o'\\ \tt guess_feedback: f<yellow> e<red> t<red> e<green> d<green> 
	  } 
	   } 
	   } 
	 & \\ 
 

    \theutterance \stepcounter{utterance}  

    & & & \multicolumn{4}{p{0.6\linewidth}}{\cellcolor[rgb]{1,0.85,0.72}{%
	\makecell[{{p{\linewidth}}}]{% 
	  \tt {\tiny [GM$\langle$B]}  
	 agreement: no\\ \tt explanation: While your guess "found" correctly places 'e' and 'd' in the fourth and fifth positions respectively, and includes 'f' in a new position, it doesn't align with the latest guess feedback. The feedback shows that 'e' is red in the second position, meaning it shouldn't be there. Also, 't' is shown as red, but your guess doesn't include 't'. Furthermore, we know from previous feedback that 
	  } 
	   } 
	   } 
	  \\ 
 

    \theutterance \stepcounter{utterance}  

    & & & \multicolumn{2}{p{0.3\linewidth}}{\cellcolor[rgb]{0.95,0.95,0.95}{%
	\makecell[{{p{\linewidth}}}]{% 
	  \tt {\tiny [GM$|$GM]}  
	 {'agreement': 'no', 'explanation': 'While your guess "found" correctly places \'e\' and \'d\' in the fourth and fifth positions respectively, and includes \'f\' in a new position, it doesn\'t align with the latest guess feedback. The feedback shows that \'e\' is red in the second position, meaning it shouldn\'t be there. Also, \'t\' is shown as red, but your guess doesn\'t include \'t\'. Furthermore, we know from previous feedback that'} 
	  } 
	   } 
	   } 
	 & & \\ 
 

    \theutterance \stepcounter{utterance}  

    & & & \multicolumn{2}{p{0.3\linewidth}}{\cellcolor[rgb]{0.95,0.95,0.95}{%
	\makecell[{{p{\linewidth}}}]{% 
	  \tt {\tiny [GM$|$GM]}  
	 Critic disagrees with the Guesser -- Sharing the critic's explanation with the guesser 
	  } 
	   } 
	   } 
	 & & \\ 
 

    \theutterance \stepcounter{utterance}  

    & & \multicolumn{4}{p{0.6\linewidth}}{\cellcolor[rgb]{0.9,0.9,0.9}{%
	\makecell[{{p{\linewidth}}}]{% 
	  \tt {\tiny [A$\langle$GM]}  
	 clue:hungry\\ \tt guess_agreement:no\\ \tt agreement_explanation:While your guess "found" correctly places 'e' and 'd' in the fourth and fifth positions respectively, and includes 'f' in a new position, it doesn't align with the latest guess feedback. The feedback shows that 'e' is red in the second position, meaning it shouldn't be there. Also, 't' is shown as red, but your guess doesn't include 't'. Furthermore, we know from previous feedback that\\ \tt  
	  } 
	   } 
	   } 
	 & \\ 
 

    \theutterance \stepcounter{utterance}  

    & \multicolumn{4}{p{0.6\linewidth}}{\cellcolor[rgb]{0.8,1,0.9}{%
	\makecell[{{p{\linewidth}}}]{% 
	  \tt {\tiny [A$\rangle$GM]}  
	 I apologize for the oversight. You're absolutely right, and I thank you for pointing that out. Let me make a more informed guess based on all the feedback we've received so far.\\ \tt \\ \tt guess:faned\\ \tt explanation:This guess aligns better with all the feedback we've received. It keeps 'f' (not in the first position), 'n' (now in the second position), 'e' in the correct fourth position, and 'd' in the 
	  } 
	   } 
	   } 
	 & & \\ 
 

    \theutterance \stepcounter{utterance}  

    & & & \multicolumn{2}{p{0.3\linewidth}}{\cellcolor[rgb]{0.95,0.95,0.95}{%
	\makecell[{{p{\linewidth}}}]{% 
	  \tt {\tiny [GM$|$GM]}  
	 {'guess': 'faned', 'explanation': "This guess aligns better with all the feedback we've received. It keeps 'f' (not in the first position), 'n' (now in the second position), 'e' in the correct fourth position, and 'd' in the"} 
	  } 
	   } 
	   } 
	 & & \\ 
 

    \theutterance \stepcounter{utterance}  

    & & & \multicolumn{2}{p{0.3\linewidth}}{\cellcolor[rgb]{0.95,0.95,0.95}{%
	\makecell[{{p{\linewidth}}}]{% 
	  \tt {\tiny [GM$|$GM]}  
	 Guesser Error: NOT_VALID_ENGLISH_WORD while parsing Player 1's (model: claude-3-5-sonnet-20240620) response 
	  } 
	   } 
	   } 
	 & & \\ 
 

    \theutterance \stepcounter{utterance}  

    & & \multicolumn{4}{p{0.6\linewidth}}{\cellcolor[rgb]{0.9,0.9,0.9}{%
	\makecell[{{p{\linewidth}}}]{% 
	  \tt {\tiny [A$\langle$GM]}  
	 your guess is not a valid word for this game. Please try again 
	  } 
	   } 
	   } 
	 & \\ 
 

    \theutterance \stepcounter{utterance}  

    & \multicolumn{4}{p{0.6\linewidth}}{\cellcolor[rgb]{0.8,1,0.9}{%
	\makecell[{{p{\linewidth}}}]{% 
	  \tt {\tiny [A$\rangle$GM]}  
	 I apologize for the invalid word. Let me try again with a valid word that fits our current information.\\ \tt \\ \tt guess:famed\\ \tt explanation:This guess incorporates all the correct information we've gathered so far. 'F' is included but not in the first position, 'e' is in the correct fourth position, and 'd' remains in the correct fifth position. We know 'n' is in the word but not in the third position, so I've move 
	  } 
	   } 
	   } 
	 & & \\ 
 

    \theutterance \stepcounter{utterance}  

    & & & \multicolumn{2}{p{0.3\linewidth}}{\cellcolor[rgb]{0.95,0.95,0.95}{%
	\makecell[{{p{\linewidth}}}]{% 
	  \tt {\tiny [GM$|$GM]}  
	 {'guess': 'famed', 'explanation': "This guess incorporates all the correct information we've gathered so far. 'F' is included but not in the first position, 'e' is in the correct fourth position, and 'd' remains in the correct fifth position. We know 'n' is in the word but not in the third position, so I've move"} 
	  } 
	   } 
	   } 
	 & & \\ 
 

    \theutterance \stepcounter{utterance}  

    & & & \multicolumn{2}{p{0.3\linewidth}}{\cellcolor[rgb]{0.95,0.95,0.95}{%
	\makecell[{{p{\linewidth}}}]{% 
	  \tt {\tiny [GM$|$GM]}  
	 Change in player1's guess\\ \tt guess_before_critic_opinion: found\\ \tt                                                 critic_agreement: no\\ \tt guess_after_critic_opinion: famed\\ \tt                                                 Proceeding with guess validation 
	  } 
	   } 
	   } 
	 & & \\ 
 

    \theutterance \stepcounter{utterance}  

    & & & \multicolumn{2}{p{0.3\linewidth}}{\cellcolor[rgb]{0.95,0.95,0.95}{%
	\makecell[{{p{\linewidth}}}]{% 
	  \tt {\tiny [GM$|$GM]}  
	 attempts: 6\\ \tt target_word = unfed\\ \tt target_word_clue = hungry\\ \tt guess: famed\\ \tt guess_feedback: f<yellow> a<red> m<red> e<green> d<green> 
	  } 
	   } 
	   } 
	 & & \\ 
 

    \theutterance \stepcounter{utterance}  

    & & & \multicolumn{2}{p{0.3\linewidth}}{\cellcolor[rgb]{0.95,0.95,0.95}{%
	\makecell[{{p{\linewidth}}}]{% 
	  \tt {\tiny [GM$|$GM]}  
	 game_result = LOSS 
	  } 
	   } 
	   } 
	 & & \\ 
 

\end{supertabular}
}

\end{document}
