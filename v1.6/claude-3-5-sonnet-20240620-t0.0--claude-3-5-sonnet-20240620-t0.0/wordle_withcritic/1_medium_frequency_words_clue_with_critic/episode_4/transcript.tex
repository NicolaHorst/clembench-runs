
\documentclass{article}
\usepackage{colortbl}
\usepackage{makecell}
\usepackage{multirow}
\usepackage{supertabular}

\begin{document}

\newcounter{utterance}

\twocolumn

{ \footnotesize  \setcounter{utterance}{1}
\setlength{\tabcolsep}{0pt}
\begin{supertabular}{c@{$\;$}|p{.15\linewidth}@{}p{.15\linewidth}p{.15\linewidth}p{.15\linewidth}p{.15\linewidth}p{.15\linewidth}}

    \# & $\;$A & \multicolumn{4}{c}{Game Master} & $\;\:$B\\
    \hline 

    \theutterance \stepcounter{utterance}  

    & & \multicolumn{4}{p{0.6\linewidth}}{\cellcolor[rgb]{0.9,0.9,0.9}{%
	\makecell[{{p{\linewidth}}}]{% 
	  \tt {\tiny [A$\langle$GM]}  
	 You are a language wizard who likes to guess words by using the given rules.\\ \tt \\ \tt Welcome to Wordle! You have six attempts to guess the target word, a valid English word of five lowercase letters (a-z). Please use the tags "guess:" and "explanation:" to provide a concise explanation for each guess.\\ \tt \\ \tt To help you make an informed guess, you will receive a clue for the word, such as\\ \tt clue: "snowy white"\\ \tt \\ \tt Here is an example guess based on the clue:\\ \tt guess: apple\\ \tt explanation: In the fairy tail Snow White, the girl is killed because she eats a poisoned apple. And the word apple has 5 letters.\\ \tt \\ \tt I will then indicate whether I agree or disagree with your guess and provide rationale, but agreeing with a guess does not confirm its correctness. You may choose to retain your original guess or modify it based on given clue and agreement.\\ \tt \\ \tt After each guess, your answer will be validated, and you will receive feedback indicating which letters are correct (green), which letters are correct but in the wrong position (yellow), and which letters are incorrect (red). This feedback can be useful in determining which letters to include or exclude in your next guess.\\ \tt \\ \tt For example, the feedback for "apple" might be:\\ \tt guess_feedback: a<yellow> p<yellow> p<green> l<yellow> e<red>\\ \tt \\ \tt Here are some rules to keep in mind:\\ \tt 1. Please do not generate any guess feedback or clue on your own\\ \tt 2. Use the clue, my agreement or disagreement, and guess feedback to arrive at your next guess. This should be reflected in your explanation as well.\\ \tt \\ \tt Let's begin with your first guess.\\ \tt \\ \tt clue:square\\ \tt  
	  } 
	   } 
	   } 
	 & \\ 
 

    \theutterance \stepcounter{utterance}  

    & \multicolumn{4}{p{0.6\linewidth}}{\cellcolor[rgb]{0.8,1,0.9}{%
	\makecell[{{p{\linewidth}}}]{% 
	  \tt {\tiny [A$\rangle$GM]}  
	 Understood. I'll use my language skills to guess the word based on the given clue and provide explanations for each guess. I'll also use any feedback provided to refine my subsequent guesses.\\ \tt \\ \tt guess: block\\ \tt explanation: A block is a three-dimensional object that often has a square face. It's a five-letter word that relates to the clue "square" as squares are often used to describe the faces of blocks or cubes. 
	  } 
	   } 
	   } 
	 & & \\ 
 

    \theutterance \stepcounter{utterance}  

    & & & \multicolumn{2}{p{0.3\linewidth}}{\cellcolor[rgb]{0.95,0.95,0.95}{%
	\makecell[{{p{\linewidth}}}]{% 
	  \tt {\tiny [GM$|$GM]}  
	 {'guess': 'block', 'explanation': 'A block is a three-dimensional object that often has a square face. It\'s a five-letter word that relates to the clue "square" as squares are often used to describe the faces of blocks or cubes.'} 
	  } 
	   } 
	   } 
	 & & \\ 
 

    \theutterance \stepcounter{utterance}  

    & & \multicolumn{4}{p{0.6\linewidth}}{\cellcolor[rgb]{0.9,0.9,0.9}{%
	\makecell[{{p{\linewidth}}}]{% 
	  \tt {\tiny [GM$\rangle$B]}  
	 I need your assistance with a word game in which we need to find a 5-letter word using a clue, a guess and an explanation for the guess.\\ \tt \\ \tt For example, if the clue is "snow white" and the guess is "apple", you will get the information like this:\\ \tt clue:snow white\\ \tt guess:apple\\ \tt explanation:the fairy tale Snow White involves the girl eating a poisoned apple, which has 5 letters.\\ \tt \\ \tt Your task is to either agree or disagree with my guess based on the given clue.\\ \tt \\ \tt If you disagree, I will make another guess, and if you agree, I will proceed in the game using the same guess.\\ \tt \\ \tt Initially, guess feedback is not available, but as the game proceeds, it becomes available. The guess feedback is not for current guess, but rather an indication of what letters can be present in the current guess. A letter highlighted in green indicates that it is a correct letter in the correct position, while yellow indicates that it is a correct letter in the wrong position, and red indicates a wrong letter. At this point, you should use all the provided information, including the clue, guess, explanation, and guess feedback, to decide whether the given guess matches the clue, and also stick to guess feedback or not.\\ \tt \\ \tt Please respond in lowercase letters and stick to this format:\\ \tt agreement:yes or no\\ \tt explanation:your reason for doing so\\ \tt \\ \tt Please note that you must not guess any word based on the riddle clue and stick to the given format while responding.\\ \tt \\ \tt Let's begin!\\ \tt \\ \tt clue:square\\ \tt guess:block\\ \tt explanation:A block is a three-dimensional object that often has a square face. It's a five-letter word that relates to the clue "square" as squares are often used to describe the faces of blocks or cubes.\\ \tt  
	  } 
	   } 
	   } 
	 & \\ 
 

    \theutterance \stepcounter{utterance}  

    & & & \multicolumn{4}{p{0.6\linewidth}}{\cellcolor[rgb]{1,0.85,0.72}{%
	\makecell[{{p{\linewidth}}}]{% 
	  \tt {\tiny [GM$\langle$B]}  
	 agreement: yes\\ \tt explanation: your guess of "block" is reasonable for the clue "square". a block often has square faces, and the word "block" is indeed five letters long. the explanation provided is logical and connects the guess to the clue effectively. 
	  } 
	   } 
	   } 
	  \\ 
 

    \theutterance \stepcounter{utterance}  

    & & & \multicolumn{2}{p{0.3\linewidth}}{\cellcolor[rgb]{0.95,0.95,0.95}{%
	\makecell[{{p{\linewidth}}}]{% 
	  \tt {\tiny [GM$|$GM]}  
	 {'agreement': 'yes', 'explanation': 'your guess of "block" is reasonable for the clue "square". a block often has square faces, and the word "block" is indeed five letters long. the explanation provided is logical and connects the guess to the clue effectively.'} 
	  } 
	   } 
	   } 
	 & & \\ 
 

    \theutterance \stepcounter{utterance}  

    & & & \multicolumn{2}{p{0.3\linewidth}}{\cellcolor[rgb]{0.95,0.95,0.95}{%
	\makecell[{{p{\linewidth}}}]{% 
	  \tt {\tiny [GM$|$GM]}  
	 Critic agrees with the Guesser -- Sharing the critic's explanation with the guesser 
	  } 
	   } 
	   } 
	 & & \\ 
 

    \theutterance \stepcounter{utterance}  

    & & \multicolumn{4}{p{0.6\linewidth}}{\cellcolor[rgb]{0.9,0.9,0.9}{%
	\makecell[{{p{\linewidth}}}]{% 
	  \tt {\tiny [A$\langle$GM]}  
	 clue:square\\ \tt guess_agreement:yes\\ \tt agreement_explanation:your guess of "block" is reasonable for the clue "square". a block often has square faces, and the word "block" is indeed five letters long. the explanation provided is logical and connects the guess to the clue effectively.\\ \tt  
	  } 
	   } 
	   } 
	 & \\ 
 

    \theutterance \stepcounter{utterance}  

    & \multicolumn{4}{p{0.6\linewidth}}{\cellcolor[rgb]{0.8,1,0.9}{%
	\makecell[{{p{\linewidth}}}]{% 
	  \tt {\tiny [A$\rangle$GM]}  
	 Thank you for agreeing with my guess. Since you've agreed, I'll maintain my original guess for this round.\\ \tt \\ \tt guess: block\\ \tt explanation: As previously stated, a block often has square faces, making it closely related to the clue "square". It's a five-letter word that fits the Wordle criteria, and the connection between blocks and squares is strong in geometric contexts. 
	  } 
	   } 
	   } 
	 & & \\ 
 

    \theutterance \stepcounter{utterance}  

    & & & \multicolumn{2}{p{0.3\linewidth}}{\cellcolor[rgb]{0.95,0.95,0.95}{%
	\makecell[{{p{\linewidth}}}]{% 
	  \tt {\tiny [GM$|$GM]}  
	 {'guess': 'block', 'explanation': 'As previously stated, a block often has square faces, making it closely related to the clue "square". It\'s a five-letter word that fits the Wordle criteria, and the connection between blocks and squares is strong in geometric contexts.'} 
	  } 
	   } 
	   } 
	 & & \\ 
 

    \theutterance \stepcounter{utterance}  

    & & & \multicolumn{2}{p{0.3\linewidth}}{\cellcolor[rgb]{0.95,0.95,0.95}{%
	\makecell[{{p{\linewidth}}}]{% 
	  \tt {\tiny [GM$|$GM]}  
	 No change in player1's guess\\ \tt guess_before_critic_opinion: block\\ \tt                                                 critic_agreement: yes\\ \tt guess_after_critic_opinion: block\\ \tt                                                 Proceeding with guess validation 
	  } 
	   } 
	   } 
	 & & \\ 
 

    \theutterance \stepcounter{utterance}  

    & & & \multicolumn{2}{p{0.3\linewidth}}{\cellcolor[rgb]{0.95,0.95,0.95}{%
	\makecell[{{p{\linewidth}}}]{% 
	  \tt {\tiny [GM$|$GM]}  
	 attempts: 1\\ \tt target_word = nerdy\\ \tt target_word_clue = square\\ \tt guess: block\\ \tt guess_feedback: b<red> l<red> o<red> c<red> k<red> 
	  } 
	   } 
	   } 
	 & & \\ 
 

    \theutterance \stepcounter{utterance}  

    & & \multicolumn{4}{p{0.6\linewidth}}{\cellcolor[rgb]{0.9,0.9,0.9}{%
	\makecell[{{p{\linewidth}}}]{% 
	  \tt {\tiny [A$\langle$GM]}  
	 guess_feedback: b<red> l<red> o<red> c<red> k<red> 
	  } 
	   } 
	   } 
	 & \\ 
 

    \theutterance \stepcounter{utterance}  

    & \multicolumn{4}{p{0.6\linewidth}}{\cellcolor[rgb]{0.8,1,0.9}{%
	\makecell[{{p{\linewidth}}}]{% 
	  \tt {\tiny [A$\rangle$GM]}  
	 Thank you for the feedback. It appears that none of the letters in "block" are correct. I'll use this information to make a new guess.\\ \tt \\ \tt guess: panes\\ \tt explanation: Given that none of the letters from "block" are in the target word, I'll try a completely different set of letters. "Panes" relates to the clue "square" as window panes are often square-shaped. It's a five-letter word that doesn't use any 
	  } 
	   } 
	   } 
	 & & \\ 
 

    \theutterance \stepcounter{utterance}  

    & & & \multicolumn{2}{p{0.3\linewidth}}{\cellcolor[rgb]{0.95,0.95,0.95}{%
	\makecell[{{p{\linewidth}}}]{% 
	  \tt {\tiny [GM$|$GM]}  
	 {'guess': 'panes', 'explanation': 'Given that none of the letters from "block" are in the target word, I\'ll try a completely different set of letters. "Panes" relates to the clue "square" as window panes are often square-shaped. It\'s a five-letter word that doesn\'t use any'} 
	  } 
	   } 
	   } 
	 & & \\ 
 

    \theutterance \stepcounter{utterance}  

    & & \multicolumn{4}{p{0.6\linewidth}}{\cellcolor[rgb]{0.9,0.9,0.9}{%
	\makecell[{{p{\linewidth}}}]{% 
	  \tt {\tiny [GM$\rangle$B]}  
	 clue:square\\ \tt guess:panes\\ \tt explanation:Given that none of the letters from "block" are in the target word, I'll try a completely different set of letters. "Panes" relates to the clue "square" as window panes are often square-shaped. It's a five-letter word that doesn't use any\\ \tt guess_feedback: b<red> l<red> o<red> c<red> k<red> 
	  } 
	   } 
	   } 
	 & \\ 
 

    \theutterance \stepcounter{utterance}  

    & & & \multicolumn{4}{p{0.6\linewidth}}{\cellcolor[rgb]{1,0.85,0.72}{%
	\makecell[{{p{\linewidth}}}]{% 
	  \tt {\tiny [GM$\langle$B]}  
	 agreement: yes\\ \tt explanation: Your guess "panes" is a good fit for the clue "square". Window panes are indeed often square-shaped, and the word "panes" is five letters long. Additionally, it doesn't contain any of the letters from the previous guess "block", which were all marked as incorrect in the guess feedback. This aligns well with the information we have so far. 
	  } 
	   } 
	   } 
	  \\ 
 

    \theutterance \stepcounter{utterance}  

    & & & \multicolumn{2}{p{0.3\linewidth}}{\cellcolor[rgb]{0.95,0.95,0.95}{%
	\makecell[{{p{\linewidth}}}]{% 
	  \tt {\tiny [GM$|$GM]}  
	 {'agreement': 'yes', 'explanation': 'Your guess "panes" is a good fit for the clue "square". Window panes are indeed often square-shaped, and the word "panes" is five letters long. Additionally, it doesn\'t contain any of the letters from the previous guess "block", which were all marked as incorrect in the guess feedback. This aligns well with the information we have so far.'} 
	  } 
	   } 
	   } 
	 & & \\ 
 

    \theutterance \stepcounter{utterance}  

    & & & \multicolumn{2}{p{0.3\linewidth}}{\cellcolor[rgb]{0.95,0.95,0.95}{%
	\makecell[{{p{\linewidth}}}]{% 
	  \tt {\tiny [GM$|$GM]}  
	 Critic agrees with the Guesser -- Sharing the critic's explanation with the guesser 
	  } 
	   } 
	   } 
	 & & \\ 
 

    \theutterance \stepcounter{utterance}  

    & & \multicolumn{4}{p{0.6\linewidth}}{\cellcolor[rgb]{0.9,0.9,0.9}{%
	\makecell[{{p{\linewidth}}}]{% 
	  \tt {\tiny [A$\langle$GM]}  
	 clue:square\\ \tt guess_agreement:yes\\ \tt agreement_explanation:Your guess "panes" is a good fit for the clue "square". Window panes are indeed often square-shaped, and the word "panes" is five letters long. Additionally, it doesn't contain any of the letters from the previous guess "block", which were all marked as incorrect in the guess feedback. This aligns well with the information we have so far.\\ \tt  
	  } 
	   } 
	   } 
	 & \\ 
 

    \theutterance \stepcounter{utterance}  

    & \multicolumn{4}{p{0.6\linewidth}}{\cellcolor[rgb]{0.8,1,0.9}{%
	\makecell[{{p{\linewidth}}}]{% 
	  \tt {\tiny [A$\rangle$GM]}  
	 guess:panes\\ \tt explanation:I'll stick with my guess of "panes" as it aligns well with the clue "square" and doesn't use any of the incorrect letters from the previous guess. Window panes are frequently square-shaped, making this a logical choice given the information we have. 
	  } 
	   } 
	   } 
	 & & \\ 
 

    \theutterance \stepcounter{utterance}  

    & & & \multicolumn{2}{p{0.3\linewidth}}{\cellcolor[rgb]{0.95,0.95,0.95}{%
	\makecell[{{p{\linewidth}}}]{% 
	  \tt {\tiny [GM$|$GM]}  
	 {'guess': 'panes', 'explanation': 'I\'ll stick with my guess of "panes" as it aligns well with the clue "square" and doesn\'t use any of the incorrect letters from the previous guess. Window panes are frequently square-shaped, making this a logical choice given the information we have.'} 
	  } 
	   } 
	   } 
	 & & \\ 
 

    \theutterance \stepcounter{utterance}  

    & & & \multicolumn{2}{p{0.3\linewidth}}{\cellcolor[rgb]{0.95,0.95,0.95}{%
	\makecell[{{p{\linewidth}}}]{% 
	  \tt {\tiny [GM$|$GM]}  
	 No change in player1's guess\\ \tt guess_before_critic_opinion: panes\\ \tt                                                 critic_agreement: yes\\ \tt guess_after_critic_opinion: panes\\ \tt                                                 Proceeding with guess validation 
	  } 
	   } 
	   } 
	 & & \\ 
 

    \theutterance \stepcounter{utterance}  

    & & & \multicolumn{2}{p{0.3\linewidth}}{\cellcolor[rgb]{0.95,0.95,0.95}{%
	\makecell[{{p{\linewidth}}}]{% 
	  \tt {\tiny [GM$|$GM]}  
	 attempts: 2\\ \tt target_word = nerdy\\ \tt target_word_clue = square\\ \tt guess: panes\\ \tt guess_feedback: p<red> a<red> n<yellow> e<yellow> s<red> 
	  } 
	   } 
	   } 
	 & & \\ 
 

    \theutterance \stepcounter{utterance}  

    & & \multicolumn{4}{p{0.6\linewidth}}{\cellcolor[rgb]{0.9,0.9,0.9}{%
	\makecell[{{p{\linewidth}}}]{% 
	  \tt {\tiny [A$\langle$GM]}  
	 guess_feedback: p<red> a<red> n<yellow> e<yellow> s<red> 
	  } 
	   } 
	   } 
	 & \\ 
 

    \theutterance \stepcounter{utterance}  

    & \multicolumn{4}{p{0.6\linewidth}}{\cellcolor[rgb]{0.8,1,0.9}{%
	\makecell[{{p{\linewidth}}}]{% 
	  \tt {\tiny [A$\rangle$GM]}  
	 guess:lined\\ \tt explanation:Based on the previous feedback, we know that 'n' and 'e' are in the word but in different positions. The word "lined" incorporates these letters in new positions and relates to the clue "square" as lines often form squares. It's a five-letter word that doesn't use any of the incorrect letters from previous guesses (b, l, o, c, k, p, a, s). 
	  } 
	   } 
	   } 
	 & & \\ 
 

    \theutterance \stepcounter{utterance}  

    & & & \multicolumn{2}{p{0.3\linewidth}}{\cellcolor[rgb]{0.95,0.95,0.95}{%
	\makecell[{{p{\linewidth}}}]{% 
	  \tt {\tiny [GM$|$GM]}  
	 {'guess': 'lined', 'explanation': 'Based on the previous feedback, we know that \'n\' and \'e\' are in the word but in different positions. The word "lined" incorporates these letters in new positions and relates to the clue "square" as lines often form squares. It\'s a five-letter word that doesn\'t use any of the incorrect letters from previous guesses (b, l, o, c, k, p, a, s).'} 
	  } 
	   } 
	   } 
	 & & \\ 
 

    \theutterance \stepcounter{utterance}  

    & & \multicolumn{4}{p{0.6\linewidth}}{\cellcolor[rgb]{0.9,0.9,0.9}{%
	\makecell[{{p{\linewidth}}}]{% 
	  \tt {\tiny [GM$\rangle$B]}  
	 clue:square\\ \tt guess:lined\\ \tt explanation:Based on the previous feedback, we know that 'n' and 'e' are in the word but in different positions. The word "lined" incorporates these letters in new positions and relates to the clue "square" as lines often form squares. It's a five-letter word that doesn't use any of the incorrect letters from previous guesses (b, l, o, c, k, p, a, s).\\ \tt guess_feedback: p<red> a<red> n<yellow> e<yellow> s<red> 
	  } 
	   } 
	   } 
	 & \\ 
 

    \theutterance \stepcounter{utterance}  

    & & & \multicolumn{4}{p{0.6\linewidth}}{\cellcolor[rgb]{1,0.85,0.72}{%
	\makecell[{{p{\linewidth}}}]{% 
	  \tt {\tiny [GM$\langle$B]}  
	 agreement: no\\ \tt explanation: While your reasoning for choosing "lined" is logical based on the previous feedback, it doesn't fully align with the current guess feedback. The feedback shows that 'n' and 'e' are correct letters but in the wrong positions (yellow). However, your guess "lined" places 'n' and 'e' in the same positions as the previous guess "panes". To better utilize the feedback, a guess should try these letters in different 
	  } 
	   } 
	   } 
	  \\ 
 

    \theutterance \stepcounter{utterance}  

    & & & \multicolumn{2}{p{0.3\linewidth}}{\cellcolor[rgb]{0.95,0.95,0.95}{%
	\makecell[{{p{\linewidth}}}]{% 
	  \tt {\tiny [GM$|$GM]}  
	 {'agreement': 'no', 'explanation': 'While your reasoning for choosing "lined" is logical based on the previous feedback, it doesn\'t fully align with the current guess feedback. The feedback shows that \'n\' and \'e\' are correct letters but in the wrong positions (yellow). However, your guess "lined" places \'n\' and \'e\' in the same positions as the previous guess "panes". To better utilize the feedback, a guess should try these letters in different'} 
	  } 
	   } 
	   } 
	 & & \\ 
 

    \theutterance \stepcounter{utterance}  

    & & & \multicolumn{2}{p{0.3\linewidth}}{\cellcolor[rgb]{0.95,0.95,0.95}{%
	\makecell[{{p{\linewidth}}}]{% 
	  \tt {\tiny [GM$|$GM]}  
	 Critic disagrees with the Guesser -- Sharing the critic's explanation with the guesser 
	  } 
	   } 
	   } 
	 & & \\ 
 

    \theutterance \stepcounter{utterance}  

    & & \multicolumn{4}{p{0.6\linewidth}}{\cellcolor[rgb]{0.9,0.9,0.9}{%
	\makecell[{{p{\linewidth}}}]{% 
	  \tt {\tiny [A$\langle$GM]}  
	 clue:square\\ \tt guess_agreement:no\\ \tt agreement_explanation:While your reasoning for choosing "lined" is logical based on the previous feedback, it doesn't fully align with the current guess feedback. The feedback shows that 'n' and 'e' are correct letters but in the wrong positions (yellow). However, your guess "lined" places 'n' and 'e' in the same positions as the previous guess "panes". To better utilize the feedback, a guess should try these letters in different\\ \tt  
	  } 
	   } 
	   } 
	 & \\ 
 

    \theutterance \stepcounter{utterance}  

    & \multicolumn{4}{p{0.6\linewidth}}{\cellcolor[rgb]{0.8,1,0.9}{%
	\makecell[{{p{\linewidth}}}]{% 
	  \tt {\tiny [A$\rangle$GM]}  
	 I apologize for the oversight. You're absolutely right, and I appreciate your correction. Let me make a new guess that better utilizes the feedback we've received.\\ \tt \\ \tt guess: tuner\\ \tt explanation: This guess places 'n' and 'e' in different positions from the previous guess, as suggested by the yellow feedback. While "tuner" doesn't directly relate to "square," it fits the letter criteria we've established. The connection to "square" could be that 
	  } 
	   } 
	   } 
	 & & \\ 
 

    \theutterance \stepcounter{utterance}  

    & & & \multicolumn{2}{p{0.3\linewidth}}{\cellcolor[rgb]{0.95,0.95,0.95}{%
	\makecell[{{p{\linewidth}}}]{% 
	  \tt {\tiny [GM$|$GM]}  
	 {'guess': 'tuner', 'explanation': 'This guess places \'n\' and \'e\' in different positions from the previous guess, as suggested by the yellow feedback. While "tuner" doesn\'t directly relate to "square," it fits the letter criteria we\'ve established. The connection to "square" could be that'} 
	  } 
	   } 
	   } 
	 & & \\ 
 

    \theutterance \stepcounter{utterance}  

    & & & \multicolumn{2}{p{0.3\linewidth}}{\cellcolor[rgb]{0.95,0.95,0.95}{%
	\makecell[{{p{\linewidth}}}]{% 
	  \tt {\tiny [GM$|$GM]}  
	 Change in player1's guess\\ \tt guess_before_critic_opinion: lined\\ \tt                                                 critic_agreement: no\\ \tt guess_after_critic_opinion: tuner\\ \tt                                                 Proceeding with guess validation 
	  } 
	   } 
	   } 
	 & & \\ 
 

    \theutterance \stepcounter{utterance}  

    & & & \multicolumn{2}{p{0.3\linewidth}}{\cellcolor[rgb]{0.95,0.95,0.95}{%
	\makecell[{{p{\linewidth}}}]{% 
	  \tt {\tiny [GM$|$GM]}  
	 attempts: 3\\ \tt target_word = nerdy\\ \tt target_word_clue = square\\ \tt guess: tuner\\ \tt guess_feedback: t<red> u<red> n<yellow> e<yellow> r<yellow> 
	  } 
	   } 
	   } 
	 & & \\ 
 

    \theutterance \stepcounter{utterance}  

    & & \multicolumn{4}{p{0.6\linewidth}}{\cellcolor[rgb]{0.9,0.9,0.9}{%
	\makecell[{{p{\linewidth}}}]{% 
	  \tt {\tiny [A$\langle$GM]}  
	 guess_feedback: t<red> u<red> n<yellow> e<yellow> r<yellow> 
	  } 
	   } 
	   } 
	 & \\ 
 

    \theutterance \stepcounter{utterance}  

    & \multicolumn{4}{p{0.6\linewidth}}{\cellcolor[rgb]{0.8,1,0.9}{%
	\makecell[{{p{\linewidth}}}]{% 
	  \tt {\tiny [A$\rangle$GM]}  
	 guess:honey\\ \tt explanation:Based on the new feedback, we now know that 'n', 'e', and 'r' are all in the word but in different positions. "Honey" incorporates these three letters in new positions. While it doesn't directly relate to "square," it fits the letter criteria we've established. The connection to "square" could be that honey is often stored in square-shaped jars or containers.\\ \tt  
	  } 
	   } 
	   } 
	 & & \\ 
 

    \theutterance \stepcounter{utterance}  

    & & & \multicolumn{2}{p{0.3\linewidth}}{\cellcolor[rgb]{0.95,0.95,0.95}{%
	\makecell[{{p{\linewidth}}}]{% 
	  \tt {\tiny [GM$|$GM]}  
	 {'guess': 'honey', 'explanation': 'Based on the new feedback, we now know that \'n\', \'e\', and \'r\' are all in the word but in different positions. "Honey" incorporates these three letters in new positions. While it doesn\'t directly relate to "square," it fits the letter criteria we\'ve established. The connection to "square" could be that honey is often stored in square-shaped jars or containers.'} 
	  } 
	   } 
	   } 
	 & & \\ 
 

    \theutterance \stepcounter{utterance}  

    & & \multicolumn{4}{p{0.6\linewidth}}{\cellcolor[rgb]{0.9,0.9,0.9}{%
	\makecell[{{p{\linewidth}}}]{% 
	  \tt {\tiny [GM$\rangle$B]}  
	 clue:square\\ \tt guess:honey\\ \tt explanation:Based on the new feedback, we now know that 'n', 'e', and 'r' are all in the word but in different positions. "Honey" incorporates these three letters in new positions. While it doesn't directly relate to "square," it fits the letter criteria we've established. The connection to "square" could be that honey is often stored in square-shaped jars or containers.\\ \tt guess_feedback: t<red> u<red> n<yellow> e<yellow> r<yellow> 
	  } 
	   } 
	   } 
	 & \\ 
 

    \theutterance \stepcounter{utterance}  

    & & & \multicolumn{4}{p{0.6\linewidth}}{\cellcolor[rgb]{1,0.85,0.72}{%
	\makecell[{{p{\linewidth}}}]{% 
	  \tt {\tiny [GM$\langle$B]}  
	 agreement: no\\ \tt explanation: While your reasoning for using the letters 'n', 'e', and 'y' is sound based on the previous feedback, there are two issues with this guess. First, the word "honey" doesn't strongly relate to the clue "square". The connection you've provided is quite weak. Second, and more importantly, the guess feedback for 'lined' showed that 'l' and 'i' are not in the word, 
	  } 
	   } 
	   } 
	  \\ 
 

    \theutterance \stepcounter{utterance}  

    & & & \multicolumn{2}{p{0.3\linewidth}}{\cellcolor[rgb]{0.95,0.95,0.95}{%
	\makecell[{{p{\linewidth}}}]{% 
	  \tt {\tiny [GM$|$GM]}  
	 {'agreement': 'no', 'explanation': 'While your reasoning for using the letters \'n\', \'e\', and \'y\' is sound based on the previous feedback, there are two issues with this guess. First, the word "honey" doesn\'t strongly relate to the clue "square". The connection you\'ve provided is quite weak. Second, and more importantly, the guess feedback for \'lined\' showed that \'l\' and \'i\' are not in the word,'} 
	  } 
	   } 
	   } 
	 & & \\ 
 

    \theutterance \stepcounter{utterance}  

    & & & \multicolumn{2}{p{0.3\linewidth}}{\cellcolor[rgb]{0.95,0.95,0.95}{%
	\makecell[{{p{\linewidth}}}]{% 
	  \tt {\tiny [GM$|$GM]}  
	 Critic disagrees with the Guesser -- Sharing the critic's explanation with the guesser 
	  } 
	   } 
	   } 
	 & & \\ 
 

    \theutterance \stepcounter{utterance}  

    & & \multicolumn{4}{p{0.6\linewidth}}{\cellcolor[rgb]{0.9,0.9,0.9}{%
	\makecell[{{p{\linewidth}}}]{% 
	  \tt {\tiny [A$\langle$GM]}  
	 clue:square\\ \tt guess_agreement:no\\ \tt agreement_explanation:While your reasoning for using the letters 'n', 'e', and 'y' is sound based on the previous feedback, there are two issues with this guess. First, the word "honey" doesn't strongly relate to the clue "square". The connection you've provided is quite weak. Second, and more importantly, the guess feedback for 'lined' showed that 'l' and 'i' are not in the word,\\ \tt  
	  } 
	   } 
	   } 
	 & \\ 
 

    \theutterance \stepcounter{utterance}  

    & \multicolumn{4}{p{0.6\linewidth}}{\cellcolor[rgb]{0.8,1,0.9}{%
	\makecell[{{p{\linewidth}}}]{% 
	  \tt {\tiny [A$\rangle$GM]}  
	 I apologize for the oversight. You're absolutely right, and I appreciate your correction. Let me make a more informed guess.\\ \tt \\ \tt guess:nerve\\ \tt explanation:This guess incorporates 'n', 'e', and 'r' in different positions from the previous guesses, as indicated by the yellow feedback. It avoids using any letters that we know are not in the word. While "nerve" doesn't directly relate to "square", it could be connected in the sense 
	  } 
	   } 
	   } 
	 & & \\ 
 

    \theutterance \stepcounter{utterance}  

    & & & \multicolumn{2}{p{0.3\linewidth}}{\cellcolor[rgb]{0.95,0.95,0.95}{%
	\makecell[{{p{\linewidth}}}]{% 
	  \tt {\tiny [GM$|$GM]}  
	 {'guess': 'nerve', 'explanation': 'This guess incorporates \'n\', \'e\', and \'r\' in different positions from the previous guesses, as indicated by the yellow feedback. It avoids using any letters that we know are not in the word. While "nerve" doesn\'t directly relate to "square", it could be connected in the sense'} 
	  } 
	   } 
	   } 
	 & & \\ 
 

    \theutterance \stepcounter{utterance}  

    & & & \multicolumn{2}{p{0.3\linewidth}}{\cellcolor[rgb]{0.95,0.95,0.95}{%
	\makecell[{{p{\linewidth}}}]{% 
	  \tt {\tiny [GM$|$GM]}  
	 Change in player1's guess\\ \tt guess_before_critic_opinion: honey\\ \tt                                                 critic_agreement: no\\ \tt guess_after_critic_opinion: nerve\\ \tt                                                 Proceeding with guess validation 
	  } 
	   } 
	   } 
	 & & \\ 
 

    \theutterance \stepcounter{utterance}  

    & & & \multicolumn{2}{p{0.3\linewidth}}{\cellcolor[rgb]{0.95,0.95,0.95}{%
	\makecell[{{p{\linewidth}}}]{% 
	  \tt {\tiny [GM$|$GM]}  
	 attempts: 4\\ \tt target_word = nerdy\\ \tt target_word_clue = square\\ \tt guess: nerve\\ \tt guess_feedback: n<green> e<green> r<green> v<red> e<red> 
	  } 
	   } 
	   } 
	 & & \\ 
 

    \theutterance \stepcounter{utterance}  

    & & \multicolumn{4}{p{0.6\linewidth}}{\cellcolor[rgb]{0.9,0.9,0.9}{%
	\makecell[{{p{\linewidth}}}]{% 
	  \tt {\tiny [A$\langle$GM]}  
	 guess_feedback: n<green> e<green> r<green> v<red> e<red> 
	  } 
	   } 
	   } 
	 & \\ 
 

    \theutterance \stepcounter{utterance}  

    & \multicolumn{4}{p{0.6\linewidth}}{\cellcolor[rgb]{0.8,1,0.9}{%
	\makecell[{{p{\linewidth}}}]{% 
	  \tt {\tiny [A$\rangle$GM]}  
	 guess:nears\\ \tt explanation:Based on the feedback, we now know that 'n', 'e', and 'r' are in the correct positions as the first three letters of the word. The word "nears" fits this pattern and uses 's' from our initial guesses, which hasn't been ruled out yet. While it doesn't directly relate to "square", it could be connected in the sense that as one "nears" a destination, they might 
	  } 
	   } 
	   } 
	 & & \\ 
 

    \theutterance \stepcounter{utterance}  

    & & & \multicolumn{2}{p{0.3\linewidth}}{\cellcolor[rgb]{0.95,0.95,0.95}{%
	\makecell[{{p{\linewidth}}}]{% 
	  \tt {\tiny [GM$|$GM]}  
	 {'guess': 'nears', 'explanation': 'Based on the feedback, we now know that \'n\', \'e\', and \'r\' are in the correct positions as the first three letters of the word. The word "nears" fits this pattern and uses \'s\' from our initial guesses, which hasn\'t been ruled out yet. While it doesn\'t directly relate to "square", it could be connected in the sense that as one "nears" a destination, they might'} 
	  } 
	   } 
	   } 
	 & & \\ 
 

    \theutterance \stepcounter{utterance}  

    & & \multicolumn{4}{p{0.6\linewidth}}{\cellcolor[rgb]{0.9,0.9,0.9}{%
	\makecell[{{p{\linewidth}}}]{% 
	  \tt {\tiny [GM$\rangle$B]}  
	 clue:square\\ \tt guess:nears\\ \tt explanation:Based on the feedback, we now know that 'n', 'e', and 'r' are in the correct positions as the first three letters of the word. The word "nears" fits this pattern and uses 's' from our initial guesses, which hasn't been ruled out yet. While it doesn't directly relate to "square", it could be connected in the sense that as one "nears" a destination, they might\\ \tt guess_feedback: n<green> e<green> r<green> v<red> e<red> 
	  } 
	   } 
	   } 
	 & \\ 
 

    \theutterance \stepcounter{utterance}  

    & & & \multicolumn{4}{p{0.6\linewidth}}{\cellcolor[rgb]{1,0.85,0.72}{%
	\makecell[{{p{\linewidth}}}]{% 
	  \tt {\tiny [GM$\langle$B]}  
	 agreement: no\\ \tt explanation: While "nears" does fit the letter pattern we've established with 'n', 'e', and 'r' in the correct positions, the guess feedback clearly shows that the last two letters 'v' and 'e' are incorrect (marked in red). This means that 'a' and 's' cannot be the last two letters of the word we're looking for. Additionally, the connection to the clue "square" 
	  } 
	   } 
	   } 
	  \\ 
 

    \theutterance \stepcounter{utterance}  

    & & & \multicolumn{2}{p{0.3\linewidth}}{\cellcolor[rgb]{0.95,0.95,0.95}{%
	\makecell[{{p{\linewidth}}}]{% 
	  \tt {\tiny [GM$|$GM]}  
	 {'agreement': 'no', 'explanation': 'While "nears" does fit the letter pattern we\'ve established with \'n\', \'e\', and \'r\' in the correct positions, the guess feedback clearly shows that the last two letters \'v\' and \'e\' are incorrect (marked in red). This means that \'a\' and \'s\' cannot be the last two letters of the word we\'re looking for. Additionally, the connection to the clue "square"'} 
	  } 
	   } 
	   } 
	 & & \\ 
 

    \theutterance \stepcounter{utterance}  

    & & & \multicolumn{2}{p{0.3\linewidth}}{\cellcolor[rgb]{0.95,0.95,0.95}{%
	\makecell[{{p{\linewidth}}}]{% 
	  \tt {\tiny [GM$|$GM]}  
	 Critic disagrees with the Guesser -- Sharing the critic's explanation with the guesser 
	  } 
	   } 
	   } 
	 & & \\ 
 

    \theutterance \stepcounter{utterance}  

    & & \multicolumn{4}{p{0.6\linewidth}}{\cellcolor[rgb]{0.9,0.9,0.9}{%
	\makecell[{{p{\linewidth}}}]{% 
	  \tt {\tiny [A$\langle$GM]}  
	 clue:square\\ \tt guess_agreement:no\\ \tt agreement_explanation:While "nears" does fit the letter pattern we've established with 'n', 'e', and 'r' in the correct positions, the guess feedback clearly shows that the last two letters 'v' and 'e' are incorrect (marked in red). This means that 'a' and 's' cannot be the last two letters of the word we're looking for. Additionally, the connection to the clue "square"\\ \tt  
	  } 
	   } 
	   } 
	 & \\ 
 

    \theutterance \stepcounter{utterance}  

    & \multicolumn{4}{p{0.6\linewidth}}{\cellcolor[rgb]{0.8,1,0.9}{%
	\makecell[{{p{\linewidth}}}]{% 
	  \tt {\tiny [A$\rangle$GM]}  
	 I apologize for my oversight. You're absolutely right, and I thank you for pointing that out. Let me make a more informed guess.\\ \tt \\ \tt guess: nerdy\\ \tt explanation: This guess maintains the correct positioning of 'n', 'e', and 'r' as the first three letters, which we know are correct. For the last two letters, I've chosen 'd' and 'y', which haven't been used or ruled out in previous guesses. While 
	  } 
	   } 
	   } 
	 & & \\ 
 

    \theutterance \stepcounter{utterance}  

    & & & \multicolumn{2}{p{0.3\linewidth}}{\cellcolor[rgb]{0.95,0.95,0.95}{%
	\makecell[{{p{\linewidth}}}]{% 
	  \tt {\tiny [GM$|$GM]}  
	 {'guess': 'nerdy', 'explanation': "This guess maintains the correct positioning of 'n', 'e', and 'r' as the first three letters, which we know are correct. For the last two letters, I've chosen 'd' and 'y', which haven't been used or ruled out in previous guesses. While"} 
	  } 
	   } 
	   } 
	 & & \\ 
 

    \theutterance \stepcounter{utterance}  

    & & & \multicolumn{2}{p{0.3\linewidth}}{\cellcolor[rgb]{0.95,0.95,0.95}{%
	\makecell[{{p{\linewidth}}}]{% 
	  \tt {\tiny [GM$|$GM]}  
	 Change in player1's guess\\ \tt guess_before_critic_opinion: nears\\ \tt                                                 critic_agreement: no\\ \tt guess_after_critic_opinion: nerdy\\ \tt                                                 Proceeding with guess validation 
	  } 
	   } 
	   } 
	 & & \\ 
 

    \theutterance \stepcounter{utterance}  

    & & & \multicolumn{2}{p{0.3\linewidth}}{\cellcolor[rgb]{0.95,0.95,0.95}{%
	\makecell[{{p{\linewidth}}}]{% 
	  \tt {\tiny [GM$|$GM]}  
	 attempts: 5\\ \tt target_word = nerdy\\ \tt target_word_clue = square\\ \tt guess: nerdy\\ \tt guess_feedback: n<green> e<green> r<green> d<green> y<green> 
	  } 
	   } 
	   } 
	 & & \\ 
 

    \theutterance \stepcounter{utterance}  

    & & & \multicolumn{2}{p{0.3\linewidth}}{\cellcolor[rgb]{0.95,0.95,0.95}{%
	\makecell[{{p{\linewidth}}}]{% 
	  \tt {\tiny [GM$|$GM]}  
	 game_result = WIN 
	  } 
	   } 
	   } 
	 & & \\ 
 

\end{supertabular}
}

\end{document}
