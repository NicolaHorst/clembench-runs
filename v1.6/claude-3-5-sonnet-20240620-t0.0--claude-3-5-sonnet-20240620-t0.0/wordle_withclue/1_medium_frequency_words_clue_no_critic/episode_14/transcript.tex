
\documentclass{article}
\usepackage{colortbl}
\usepackage{makecell}
\usepackage{multirow}
\usepackage{supertabular}

\begin{document}

\newcounter{utterance}

\twocolumn

{ \footnotesize  \setcounter{utterance}{1}
\setlength{\tabcolsep}{0pt}
\begin{supertabular}{c@{$\;$}|p{.15\linewidth}@{}p{.15\linewidth}p{.15\linewidth}p{.15\linewidth}p{.15\linewidth}p{.15\linewidth}}

    \# & $\;$A & \multicolumn{4}{c}{Game Master} & $\;\:$B\\
    \hline 

    \theutterance \stepcounter{utterance}  

    & & \multicolumn{4}{p{0.6\linewidth}}{\cellcolor[rgb]{0.9,0.9,0.9}{%
	\makecell[{{p{\linewidth}}}]{% 
	  \tt {\tiny [A$\langle$GM]}  
	 You are a language wizard who likes to guess words by using the given rules.\\ \tt \\ \tt Welcome to Wordle! You have six attempts to guess the target word, a valid English word of five lowercase letters (a-z). Please use the tags "guess:" and "explanation:" to provide a concise explanation for each guess.\\ \tt \\ \tt  To help you make an informed guess, you will receive a clue for the word, such as\\ \tt clue: snowy white.\\ \tt \\ \tt Here is an example guess based on the clue:\\ \tt guess: apple\\ \tt explanation: In the fairy tail Snow White, the girl is killed because she eats a poisoned apple. And the word apple has 5 letters.\\ \tt \\ \tt After each guess, your answer will be validated, and you will receive feedback indicating which letters are correct (green), which letters are correct but in the wrong position (yellow), and which letters are incorrect (red). This feedback can be useful in determining which letters to include or exclude in your next guess.\\ \tt \\ \tt For example, the feedback for "apple" might be:\\ \tt guess_feedback: a<yellow> p<yellow> p<green> l<yellow> e<red>\\ \tt \\ \tt The explanation should contain details about how both the guess_feedback and clue work together in tandem to formulate a new guess.\\ \tt \\ \tt Let's begin with your first guess.\\ \tt clue:bravado\\ \tt  
	  } 
	   } 
	   } 
	 & \\ 
 

    \theutterance \stepcounter{utterance}  

    & \multicolumn{4}{p{0.6\linewidth}}{\cellcolor[rgb]{0.8,1,0.9}{%
	\makecell[{{p{\linewidth}}}]{% 
	  \tt {\tiny [A$\rangle$GM]}  
	 Understood. I'll start guessing based on the clue "bravado" and provide explanations for each guess.\\ \tt \\ \tt guess: brave\\ \tt explanation: The word "brave" is directly related to the clue "bravado," which refers to a bold manner or show of courage. "Brave" has five letters and captures the essence of the clue. 
	  } 
	   } 
	   } 
	 & & \\ 
 

    \theutterance \stepcounter{utterance}  

    & & & \multicolumn{2}{p{0.3\linewidth}}{\cellcolor[rgb]{0.95,0.95,0.95}{%
	\makecell[{{p{\linewidth}}}]{% 
	  \tt {\tiny [GM$|$GM]}  
	 {'guess': 'brave', 'explanation': 'The word "brave" is directly related to the clue "bravado," which refers to a bold manner or show of courage. "Brave" has five letters and captures the essence of the clue.'} 
	  } 
	   } 
	   } 
	 & & \\ 
 

    \theutterance \stepcounter{utterance}  

    & & & \multicolumn{2}{p{0.3\linewidth}}{\cellcolor[rgb]{0.95,0.95,0.95}{%
	\makecell[{{p{\linewidth}}}]{% 
	  \tt {\tiny [GM$|$GM]}  
	 attempts: 1\\ \tt target_word = swash\\ \tt target_word_clue = bravado\\ \tt guess: brave\\ \tt guess_feedback: b<red> r<red> a<green> v<red> e<red> 
	  } 
	   } 
	   } 
	 & & \\ 
 

    \theutterance \stepcounter{utterance}  

    & & \multicolumn{4}{p{0.6\linewidth}}{\cellcolor[rgb]{0.9,0.9,0.9}{%
	\makecell[{{p{\linewidth}}}]{% 
	  \tt {\tiny [A$\langle$GM]}  
	 guess_feedback: b<red> r<red> a<green> v<red> e<red> 
	  } 
	   } 
	   } 
	 & \\ 
 

    \theutterance \stepcounter{utterance}  

    & \multicolumn{4}{p{0.6\linewidth}}{\cellcolor[rgb]{0.8,1,0.9}{%
	\makecell[{{p{\linewidth}}}]{% 
	  \tt {\tiny [A$\rangle$GM]}  
	 Thank you for the feedback. Let's analyze this and make a new guess.\\ \tt \\ \tt guess: salty\\ \tt explanation: Based on the previous feedback, we know that the middle letter 'a' is correct and in the right position. The other letters (b, r, v, e) are not in the word. The clue "bravado" suggests confidence or boldness, which could be associated with a sharp or strong taste. "Salty" fits this description, 
	  } 
	   } 
	   } 
	 & & \\ 
 

    \theutterance \stepcounter{utterance}  

    & & & \multicolumn{2}{p{0.3\linewidth}}{\cellcolor[rgb]{0.95,0.95,0.95}{%
	\makecell[{{p{\linewidth}}}]{% 
	  \tt {\tiny [GM$|$GM]}  
	 {'guess': 'salty', 'explanation': 'Based on the previous feedback, we know that the middle letter \'a\' is correct and in the right position. The other letters (b, r, v, e) are not in the word. The clue "bravado" suggests confidence or boldness, which could be associated with a sharp or strong taste. "Salty" fits this description,'} 
	  } 
	   } 
	   } 
	 & & \\ 
 

    \theutterance \stepcounter{utterance}  

    & & & \multicolumn{2}{p{0.3\linewidth}}{\cellcolor[rgb]{0.95,0.95,0.95}{%
	\makecell[{{p{\linewidth}}}]{% 
	  \tt {\tiny [GM$|$GM]}  
	 attempts: 2\\ \tt target_word = swash\\ \tt target_word_clue = bravado\\ \tt guess: salty\\ \tt guess_feedback: s<green> a<yellow> l<red> t<red> y<red> 
	  } 
	   } 
	   } 
	 & & \\ 
 

    \theutterance \stepcounter{utterance}  

    & & \multicolumn{4}{p{0.6\linewidth}}{\cellcolor[rgb]{0.9,0.9,0.9}{%
	\makecell[{{p{\linewidth}}}]{% 
	  \tt {\tiny [A$\langle$GM]}  
	 guess_feedback: s<green> a<yellow> l<red> t<red> y<red> 
	  } 
	   } 
	   } 
	 & \\ 
 

    \theutterance \stepcounter{utterance}  

    & \multicolumn{4}{p{0.6\linewidth}}{\cellcolor[rgb]{0.8,1,0.9}{%
	\makecell[{{p{\linewidth}}}]{% 
	  \tt {\tiny [A$\rangle$GM]}  
	 guess:stand\\ \tt explanation:From the previous feedback, we now know that 's' is correct and in the right position at the beginning of the word, and 'a' is in the word but not in the second position. The clue "bravado" suggests a confident or bold stance, which aligns with the word "stand". This word keeps the correct 's' at the start, moves 'a' to a different position, and introduces new letters to replace 
	  } 
	   } 
	   } 
	 & & \\ 
 

    \theutterance \stepcounter{utterance}  

    & & & \multicolumn{2}{p{0.3\linewidth}}{\cellcolor[rgb]{0.95,0.95,0.95}{%
	\makecell[{{p{\linewidth}}}]{% 
	  \tt {\tiny [GM$|$GM]}  
	 {'guess': 'stand', 'explanation': 'From the previous feedback, we now know that \'s\' is correct and in the right position at the beginning of the word, and \'a\' is in the word but not in the second position. The clue "bravado" suggests a confident or bold stance, which aligns with the word "stand". This word keeps the correct \'s\' at the start, moves \'a\' to a different position, and introduces new letters to replace'} 
	  } 
	   } 
	   } 
	 & & \\ 
 

    \theutterance \stepcounter{utterance}  

    & & & \multicolumn{2}{p{0.3\linewidth}}{\cellcolor[rgb]{0.95,0.95,0.95}{%
	\makecell[{{p{\linewidth}}}]{% 
	  \tt {\tiny [GM$|$GM]}  
	 attempts: 3\\ \tt target_word = swash\\ \tt target_word_clue = bravado\\ \tt guess: stand\\ \tt guess_feedback: s<green> t<red> a<green> n<red> d<red> 
	  } 
	   } 
	   } 
	 & & \\ 
 

    \theutterance \stepcounter{utterance}  

    & & \multicolumn{4}{p{0.6\linewidth}}{\cellcolor[rgb]{0.9,0.9,0.9}{%
	\makecell[{{p{\linewidth}}}]{% 
	  \tt {\tiny [A$\langle$GM]}  
	 guess_feedback: s<green> t<red> a<green> n<red> d<red> 
	  } 
	   } 
	   } 
	 & \\ 
 

    \theutterance \stepcounter{utterance}  

    & \multicolumn{4}{p{0.6\linewidth}}{\cellcolor[rgb]{0.8,1,0.9}{%
	\makecell[{{p{\linewidth}}}]{% 
	  \tt {\tiny [A$\rangle$GM]}  
	 guess:swank\\ \tt explanation:We've learned that the word starts with 's' and has 'a' as the third letter, both in the correct positions. The clue "bravado" suggests a showy or ostentatious display of confidence, which aligns well with the word "swank". This word means stylish or luxurious in a showy way, often associated with a bold or confident attitude. It maintains the correct 's' and  
	  } 
	   } 
	   } 
	 & & \\ 
 

    \theutterance \stepcounter{utterance}  

    & & & \multicolumn{2}{p{0.3\linewidth}}{\cellcolor[rgb]{0.95,0.95,0.95}{%
	\makecell[{{p{\linewidth}}}]{% 
	  \tt {\tiny [GM$|$GM]}  
	 {'guess': 'swank', 'explanation': 'We\'ve learned that the word starts with \'s\' and has \'a\' as the third letter, both in the correct positions. The clue "bravado" suggests a showy or ostentatious display of confidence, which aligns well with the word "swank". This word means stylish or luxurious in a showy way, often associated with a bold or confident attitude. It maintains the correct \'s\' and'} 
	  } 
	   } 
	   } 
	 & & \\ 
 

    \theutterance \stepcounter{utterance}  

    & & & \multicolumn{2}{p{0.3\linewidth}}{\cellcolor[rgb]{0.95,0.95,0.95}{%
	\makecell[{{p{\linewidth}}}]{% 
	  \tt {\tiny [GM$|$GM]}  
	 attempts: 4\\ \tt target_word = swash\\ \tt target_word_clue = bravado\\ \tt guess: swank\\ \tt guess_feedback: s<green> w<green> a<green> n<red> k<red> 
	  } 
	   } 
	   } 
	 & & \\ 
 

    \theutterance \stepcounter{utterance}  

    & & \multicolumn{4}{p{0.6\linewidth}}{\cellcolor[rgb]{0.9,0.9,0.9}{%
	\makecell[{{p{\linewidth}}}]{% 
	  \tt {\tiny [A$\langle$GM]}  
	 guess_feedback: s<green> w<green> a<green> n<red> k<red> 
	  } 
	   } 
	   } 
	 & \\ 
 

    \theutterance \stepcounter{utterance}  

    & \multicolumn{4}{p{0.6\linewidth}}{\cellcolor[rgb]{0.8,1,0.9}{%
	\makecell[{{p{\linewidth}}}]{% 
	  \tt {\tiny [A$\rangle$GM]}  
	 guess:swash\\ \tt explanation:We now know that the first three letters are correct: 'swa'. The clue "bravado" strongly suggests a word related to bold or flamboyant behavior. "Swash" fits perfectly with this theme, as it means to move with a flourish or swagger, which is very much in line with the idea of bravado. It keeps the correct 'swa' start and introduces new letters 's' an 
	  } 
	   } 
	   } 
	 & & \\ 
 

    \theutterance \stepcounter{utterance}  

    & & & \multicolumn{2}{p{0.3\linewidth}}{\cellcolor[rgb]{0.95,0.95,0.95}{%
	\makecell[{{p{\linewidth}}}]{% 
	  \tt {\tiny [GM$|$GM]}  
	 {'guess': 'swash', 'explanation': 'We now know that the first three letters are correct: \'swa\'. The clue "bravado" strongly suggests a word related to bold or flamboyant behavior. "Swash" fits perfectly with this theme, as it means to move with a flourish or swagger, which is very much in line with the idea of bravado. It keeps the correct \'swa\' start and introduces new letters \'s\' an'} 
	  } 
	   } 
	   } 
	 & & \\ 
 

    \theutterance \stepcounter{utterance}  

    & & & \multicolumn{2}{p{0.3\linewidth}}{\cellcolor[rgb]{0.95,0.95,0.95}{%
	\makecell[{{p{\linewidth}}}]{% 
	  \tt {\tiny [GM$|$GM]}  
	 attempts: 5\\ \tt target_word = swash\\ \tt target_word_clue = bravado\\ \tt guess: swash\\ \tt guess_feedback: s<green> w<green> a<green> s<green> h<green> 
	  } 
	   } 
	   } 
	 & & \\ 
 

    \theutterance \stepcounter{utterance}  

    & & & \multicolumn{2}{p{0.3\linewidth}}{\cellcolor[rgb]{0.95,0.95,0.95}{%
	\makecell[{{p{\linewidth}}}]{% 
	  \tt {\tiny [GM$|$GM]}  
	 game_result = WIN 
	  } 
	   } 
	   } 
	 & & \\ 
 

\end{supertabular}
}

\end{document}
